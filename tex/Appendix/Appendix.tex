\environment ../sty/sr-style
\environment ../sty/sr-essay-style
\startcomponent Appendix
\product ../main/main
%Initialize abbreviations used in this book:
\abbreviation{AIL}{Ancient Israel and Its Literature}
\abbreviation{ANEM}{Ancient Near East Monographs}
\abbreviation{ABRL}{Anchor Yale Bible Reference Library}
\abbreviation[BHS]{\italic{BHS}}{\italic{Biblica Hebraica Stuttgartensia}}
\abbreviation[BHQ]{\italic{BHQ}}{\italic{Biblica Hebraica Quinta}}
\abbreviation{BRLA}{Brill Reference Library of Judaism}
\abbreviation{CBET}{Contributions to Biblical Exegesis and Theology}
\abbreviation{FAT}{Forschungen zum Alten Testament}
\abbreviation[HALOT]{\italic{HALOT}}{\italic{The Hebrew and Aramaic Lexicon of the Old Testament}. Ludwig Koehler and Walter Baumgartner. Translated and edited under the supervision of Mervyn E. J. Richardson. 2 vols. Leiden: Brill, 2001.}
\abbreviation{HB}{Hebrew Bible}
\abbreviation{HSM}{Harvard Semitic Monographs}
\abbreviation[JAOS]{\italic{JAOS}}{\italic{Journal of the American Oriental Society}}
\abbreviation[Leningrad]{L}{Leningrad Codex (National Library of Russia Firkovich B19A)}
\abbreviation{LSAWS}{Linguistic Studies in Ancient West Semitic}
\abbreviation{MPIL}{Monographs of the Peshitta Institute Leiden}
\abbreviation{MRTS}{Medieval and Renaissance Texts and Studies}
\abbreviation{MS}{manuscript}
\abbreviation{MSS}{manuscripts}
\abbreviation{MT}{Masoretic Text}
\abbreviation{NAC}{New American Commentary}
\abbreviation{NICOT}{New International Commentary on the Old Testament}
\abbreviation{OG}{Old Greek}
\abbreviation{SANER}{Studies in Ancient Near Eastern Records}
\abbreviation{SBLDS}{Society of Biblical Literature Dissertation Series}
\abbreviation{SBLSCS}{SBL Septuagint and Cognate Studies}
\abbreviation{SR}{Siglum indicating readings adopted in this edition}
\abbreviation{STDJ}{Studies on the Texts of the Desert of Judah}
\abbreviation[TC]{\italic{TC}}{\italic{TC: A Journal of Biblical Textual Criticism}}
\abbreviation{TCST}{Text-Critical Studies}
\abbreviation[THB]{\italic{THB}}{\italic{Textual History of the Bible}}
\abbreviation{THBSup}{Supplements to \italic{Textual History of the Bible}}
\abbreviation{TOTC}{Tyndale Old Testament Commentaries}
\abbreviation[VT]{\italic{VT}}{\italic{Vetus Testamentum}}
\abbreviation{VTSup}{Supplements to \italic{Vetus Testamentum}}
\abbreviation{WLC}{Westminster Leningrad Codex}
\starttext
\startEssay[title={Appendix}]
This appendix sketches of the text-critical theory and praxis underlying this edition. (For an orientation to the edition as a whole, see the introduction provided in volume 1.) The first section will overview the history of the transmission of the text; the second will turn to praxis, outlining of some of the general principles on which text-critical decisions have been made; and the third will comment on the specific textual situation in each book of the \infull{HB} (hereafter abbreviated \HB{}). A full defense of the positions taken here and interaction with other approaches is beyond the scope of this document; the aim of the following pages is more descriptive than persuasive in form.\par
\startsection[reference=sec:history, title={An Interpretive History of the Textual Transmission of the Old Testament}]
In the view of this editor, any theory must base itself primarily on an assessment of the history of the transmission of the text, especially as documented in manuscripts (hereafter, \MSS{}; singular \MS{}) and in ancient versions. Accordingly, the following sections will first discuss this transmission history, beginning with the most ancient times (\in{§}{}[subsec:first-temple-period]), then working forwards in time, first to the Second Temple period (\in{§}{}[subsec:second-temple-period]), then to late antiquity and the medieval period (\in{§}{}[subsec:late-antiquity]).\par
\startsubsection[reference=subsec:first-temple-period, title={First Temple Period and Earlier
Periods}]
Unsurprisingly, no \MSS{} have been found from the First Temple period or earlier.\footnote{Two amulets probably date to the former period, however. See Armin Lange and Emanuel Tov, eds., \italic{The Hebrew Bible: Pentateuch, Former and Latter Prophets}, vol. 1B of \THB{} (Leiden: Brill, 2016), 115\textsuperscript{b}–118\textsuperscript{b}.} Consequently, little can be said about the early history of the text.\par Little about the mechanics of \MS{} production and dissemination is clear. While the use of tablets certainly was known in ancient Israel, the single greatest influence on ancient Israelite scribal practices would seem to be Egyptian,\footnote{Cf. David M. Carr, \italic{Writing on the Tablet of the Heart: Origins of Scripture and Literature} (Oxford: Oxford University Press, 2005), 85–90.} and this tradition (excepting monuments) employed perishable writing materials and ink. No major transfer of the \HB{} text from one medium to another or from a cuneiform to an alphabetic script, then, is likely. Phenomena within the \HB{} such as citations of one book in another and reported speech show that reproduction of minute elements of verbiage was not carried to obsessive levels in these time periods. At the same time, this very disinterest suggests a scribal culture where constant or drastic intervention in the text was not felt to be necessary. It is safe to assume that there was a general fidelity in the transmission of the text in and before the First Temple period. Given the similarity of certain letters to one another in the ancient scripts (for example, \italic{mem} and \italic{nun}), and the lack of diacritics during these periods, transcriptional error likely accounts for many of the discrepancies among ancient \MSS{} that did emerge.\par
The textual situation entering the Second Temple period likely did not fully reflect earlier situations, given the devastation wrought the Babylonian and Assyrian invasions and by the actions of King Manasseh, during whose reign Scripture, or at any rate a portion thereof, evidently had to be kept hidden, or fell into utter disuse (cf. 2 Kgs 22:8). Yet there are indications that the Levites and priests were those most associated with the dissemination of the text (cf. Lev 10:11; Deut 17:9, 18; 21:5; 2 Chr 17:7–9; 19:8), and since redistribution of the biblical books after the death of Mannasseh apparently began in association with the Temple precinct, the basis for subsequent copying was probably very pure.\par
\stopsubsection
\startsubsection[reference=subsec:second-temple-period, title={Second Temple
Period}]
While it would not be accurate to characterize the textual situation before the destruction of the Second Temple as chaotic,\footnote{Lange estimates that for the Pentateuch, 52.5\letterpercent{} of the Dead Sea Scrolls corpus is best characterized as \quotation{non-aligned,} 5\letterpercent{} as \quotation{pre-{\sc sp},} 5\letterpercent{} as \quotation{\italic{Vorlage} of {\sc lxx},} 27.5\letterpercent{} as \quotation{equally close to {\sc mt} and {\sc sp},} 5\letterpercent{} as \quotation{semi-{\sc mt},} 5\letterpercent{} as \quotation{proto-{\sc mt}}; for the other \HB{} books he sees 51\letterpercent{} of this corpus as \quotation{non-aligned,} 4\letterpercent{} as \quotation{\italic{Vorlage} of {\sc lxx},} 35\letterpercent{} as \quotation{semi-{\sc mt},} and 10\letterpercent{} as \quotation{proto-{\sc mt}} (\quotation{A Critical Edition of the Hebrew Bible between the Dead Sea Scrolls to the Masoretic Text,} in \italic{The Text of the Hebrew Bible and Its Editions: Studies in Celebration of the Fifth Centennial of the Complutensian Polyglot}, ed. Andrés Piquer Otero and Pablo A. Torijano Morales, \THBSup{} 1 [Leiden: Brill, 2016], 107–142, esp. 109; capitalization has been adapted). This is a significant level of diversity, yet as far the broad strokes are concerned, his \quotation{semi-{\sc mt}} and \quotation{proto-{\sc mt}} can probably be collapsed into a single group, and much of the \quotation{non-aligned group} surely is such simply due to the small amount of extant text. Then, too, the Samaritan Pentateuch does not differ radically from the \infull{MT}, such that roughly a third of the Pentateuch \MSS{} more or less resemble the \MT{}. Meanwhile, close to half the \MSS{} of the other books closely resemble the \MT{}.\par
In any case, only rarely is the divergence between any two \MSS{} of a given book on the order of that between the \italic{Vorlage} of the \OG{} and the \MT{} in the book of Jeremiah, and even in that case the fundamental messages of the book are hardly altered.} tight controls on the copying process appear not to have been in place.\footnote{Such controls are often posited as the explanation for the dominance of the \MT{} subsequent to this period, but the political fragmentation of the late Second Temple period makes this explanation unlikely. Regarding the possibility the \MT{} emerging from Temple-based activity specifically, see David Andrew Teeter, \italic{Scribal Laws: Exegetical Variation in the Textual Transmission of Biblical Law in the Late Second Temple Period}, \FAT{} 92 (Tübingen: Mohr Siebeck, 2014) 231–233.\par
Tov (\quotation{The Socio-Religious Setting of the (Proto-)Masoretic Text,} \italic{Textus} 27.1 [2018], 135–153, esp. 151–152) claims that there is a strong association between sectarian groups and texts unrelated to the \MT{} on the one hand and the \MT{} and emerging Rabbinic Judaism on the other. On the first association see the previous footnote. The second association is stronger than the first, yet is undercut by the presently limited state of our knowledge concerning Second Temple synagogues (see especially Lidia D. Matassa, \italic{Invention of the First-Century Synagogue}, ed. Jason M. Silverman and J. Murray Watson, \ANEM{} 22 [Atlanta, GA: SBL Press, 2018]) in general and the apparent mistaken identification of a structure as a synagogue at Masada (ibid., 109–157) in particular.} Some \MSS{} explicate as well as transmit Scripture, or even \quotation{rework} the text. Others are highly conservative. The latter group can, at least in general, be identified with the M-text. This conservative text is attested in various \MSS{}, most extremely fragmentary, from Qumran and elsewhere in the Judean wilderness. The non-M texts, for their part, also found at Qumran, appear in some cases to be ancestors of, or members of the same textual stream as, either the Samaritan Pentateuch or \infull{OG} (hereafter \OG{}) versions; in other cases, the alignment of these non-conservative \MSS{} is unclear, chiefly owing to their fragmentary state. Allowing for widely varying members, we may provisionally posit essentially two branches, one conservative and one interventionist, of text for the Second Temple period. (This non-M branch will here be labeled the G text, even though not all of its members are extant only in Greek-language sources, and not all Greek-language sources attest to this form of the text.) The ultimate genesis of these branches is at present unknown, but it seems best to see them as \quotation{complementary} in the eyes of the late Second Temple period.\footnote{The term is Teeter's (\italic{Scribal Laws}, 256).} It is worth quoting David Andrew Teeter at length here:
\startblockquote
	The perdurance of \quotation{conservative} texts of the Hebrew Bible in the face of the active and continuous scribal intervention so evident in the [late Second Temple] period … must be explained … [T]he simultaneous \italic{coexistence} of both kinds of manuscripts in a single environment, the functional \italic{difference} between the two attested by their distribution, and the \italic{structured and functional similarities} to the classical Targumim, combine to suggest that the pluriformity of Hebrew manuscripts in the late Second Temple period, a basic feature of which is a distinction between two model of transmission (exact conservatism on the one hand, and facilitating intervention on the other), is not reflective of incompatible scribal mindsets, nor does it reflect the textual practice of mutually isolated or antagonistic communities; rather, it represents the complex reality of a multiform, multi-generic textual polysystem of scriptural study, and is the expression of a monolingual, analogical hermeneutics.\footnote{\italic{Scribal Laws}, 250, 264.}
\stopblockquote

Scribal conservatism and scribal interventionism, then, coexisted, but not generally in the same lines of transmission. Yet scribal conservatism does not mean perfection. Individual scribes did their works with degrees of precision ranging from exceedingly careful to notably careless, and even the best copyists err from time to time.\par For the most part, individual books of the Bible appear on separate \MSS{}, not as parts of larger corpora.\par
Diacritic signs did not exist during this period (see §2.5 in the introduction of volume 1), leaving various ambiguities in the text and much potential for confusion (especially given the similarity, noted above, of certain letters to one another). Moreover, in the post-exilic period a new script came into vogue, such that many \MSS{} had to be transcribed from exemplars with a script different from that which the scribe producing the new \MS{} used, a process which can be expected to have provoked a certain number of errors.\par
The Maccabean revolt led to tremendous upheaval,\footnote{Cf. especially 1 Macc 1:56–57 and 3:48.} as did the first Jewish revolt against Rome, so it is not unreasonable to posit that normal transmissional patterns saw severe disruption. In particular it is to be noted that those Hebrew-language \MSS{} which remained intact after the fall of Jerusalem to Titus to form the germ of the later \MT{} might not have been wholly representative of the type of text to which they belonged. On the other hand, the situation should not be exaggerated, for many segments of the medieval text clearly evidence great antiquity.\footnote[note:evidence-great-antiquity]{As examples: (1) At various points the \HB{} preserves Egyptian words in a form that predates the Hasmonean era (cf. Yoshiyuki Muchiki, \italic{Egyptian Proper Names and Loanwords in North-West Semitic}, \SBLDS{} 173 [Atlanta, GA: Scholars Press, 1993], 324–325, and further Benjamin J. Noonan, \italic{Non-Semitic Loanwords in the Hebrew Bible: A Lexicon of Language Contact}, \LSAWS{} 14 [Winona Lake, IN: Eisenbrauns, 2019], 301–307). (2) Hebrew pronunciation in Judea around the time of the Second Jewish Revolt differed substantially from that assumed by many of the spellings of the \MT{} (see, for example, Michael Owen Wise, \italic{Language and Literacy in Roman Judaea: A Study of the Bar Kokhba Document}, \ABRL{} [New Haven, CT: Yale University Press, 2015], 260–264); the \MT{} in these cases retains the older spellings. (3) Regnal data have been preserved in the MT with a very high, even pristine, degree of accuracy (see K. A. Kitchen, \italic{On the Reliability of the Old Testament} [Grand Rapids, MI: Eerdmans, 2003], 29).}\par
\stopsubsection
\startsubsection[reference=subsec:late-antiquity, title={Late Antiquity and Medieval
Period}]
In late antiquity and the medieval period, intricate scribal conventions gradually emerged which placed a high premium on conserving the Old Testament text as then found in certain copies. These conventions concerned themselves with minute aspects of the text, such as enumerating the various spellings of a given word that appear in different passages. It even appears that in certain cases a reading (called a \italic{qerey}) was for one reason or another preferred over that of an exemplar (called a \italic{kethib}), yet the preferred reading was kept in the margin, not promoted to the main text (apart from indication of the matter by means of paratextual signs).\footnote{This matter is debated and uncertain. For a concise overview of scholarly debate on the matter, see Michael Graves, \quotation{The Origin of \italic{Ketiv-Qere} Readings,} \TC{} 8 (2003).} Thus the \MT{} can confidently be said largely to reflect an ancient type of text. Such is confirmed by the close resemblance between the \MT{} and (1) certain \MSS{} antedating the destruction of the Second Temple, as well as (2) the texts evidently underlying such translations as the Latin Vulgate and the Syriac Peshitta, versions known to have been produced in late antiquity, (3) and certain efforts to revise the \OG{} versions (on which versions see below). Also very close to the \MT{} are the Targumim, Aramaic translations. The Targumim were often used alongside original-language \MSS{}, and to varying degrees they interpret as well as transmit the text, through additions and non-literal renderings. That said, the \MT{} is not likely identical in every respect to those antecedents which were produced in the late Second Temple period, given the transmissional upheaval inherent in the many Jewish experiences of property loss and forced migration.\footnote{Examples include the Second Jewish Revolt, though the effects of that event should not be exaggerated (cf. Menahem Mor, \italic{The Second Jewish Revolt: The Bar Kokhba War, 132–136 CE}, \BRLA{} 70 [Leiden: Brill, 2016], 468–485); the Byzantine-Sasanian wars; the Islamic conquest; and the Crusades.} Indeed, certain books in the MT contain various cancellation marks and other Second Temple-era scribal signs the significance of which had become unknown by the time of the medieval Masoretes.\footnote{See Emanuel Tov, \italic{Textual Criticism of the Hebrew Bible}, 3\textsuperscript{rd} ed. (Minneapolis, MN: Fortress Press, 2012), 52–53 and 203–204.} The presence of these signs suggests that at least the books that contain them\footnote{i.e., Genesis, Numbers, Deuteronomy, Judges, Samuel, Isaiah, Ezekiel, Psalms, and Job. Whether these marks in the \MT{} signify a copying error, comparison with another \MS{}, or conjecture must be examined on a case-by-case basis.} descend from a single Second Temple-era \MS{}, the accuracy of which \MS{} can be expected to vary from book to book. Thus, a distinction might be made, albeit infrequently, between the \quotation{proto-M} (hereafter, \quotation{M}) text and the \MT{}.\par
Concurrent with the \MT{} is the transmission of the Samaritan Pentateuch. This text, similar overall to the \MT{}, contains a small number of heterodox readings, and, discoveries at Qumran have shown,\footnote{See Tov, \italic{Textual Criticism of the Hebrew Bible}, 90–93.} a number of divergences from the \MT{} which trace to Second Temple times.\footnote{The modern Samaritan Pentateuch might ultimately descend from a \MS{} from the First Temple period (cf. 2 Kgs 17:26–28). A later source, however, is more likely (Armin Lange and Emanuel Tov, eds., \italic{The Hebrew Bible: Overview Articles}, vol. 1A of \THB{} (Leiden: Brill, 2016), 168\textsuperscript{b}–169\textsuperscript{b}.} Given the widening gulf between the Jewish and Samaritans communities during late antiquity and beyond, the Samaritan Pentateuch is valuable as an independent witness to an early form of the \HB{} text.\par
Also concurrent with the \MT{} are various translations, known collectively as the Septuagint (LXX) or \infull{OG} (again, \OG{}), anciently made from the original Hebrew and Aramaic into Greek.\footnote{For an accessible introduction to the \OG{} and related subjects, see Karen H. Jobes and Moisés Silva, \italic{Invitation to the Septuagint}, 2\textsuperscript{nd} ed. (Grand Rapids, MI: Baker Academic, 2015).} The texts underlying these translations vary in their character, but frequently offer readings at variance with the \MT{}.\par
The \MSS{} from this period by no means always contain the entire \HB{}, but often include more or less defined corpora (e.g., the latter prophets, viz. Isaiah, Jeremiah, Ezekiel, and the Book of the Twelve).\par Particularly prior to the insertion of diacritical signs, confusion of similar letters (for example, \italic{waw} and \italic{yodh} in certain scribal hands, and \italic{daleth} and \italic{resh} in nearly all hands) prompted many of the errors that have come about in the transmission of the text.
\stopsubsection
\stopsection
\startsection[reference=sec:praxis, title={Praxis}]
There are, then, at least two ancient branches in the textual transmission of the \HB{} dating at least to the Second Temple period, branches labeled here M and G. In general, M is preferable for three reasons. (1) In most units of variation, M contains—as can reasonably be ascertained on intrinsic, linguistic, and other grounds—the best available reading. (2) As noted above, M appears largely, or possibly entirely, to represent the conservatively copied textual tradition, rather than the facilitating texts produced for study; by the same token, sectarian readings in M, if such exist, are difficult to detect.\footnote{See, for example Eugene Ulrich, \italic{The Dead Sea Scrolls and the Developmental Composition of the Bible}, \VTSup{} 169 (Leiden: Brill, 2015), 185.} (3) The limited extent of the Samaritan Pentateuch, fragmentary state of the Qumran fragments, and translational nature of the Greek versions make a basic reliance on these sources problematic.\par
It might be argued, primarily on internal grounds, that the assessment of the relationship of M to G just advanced should in the case of certain books be reversed. The book of Jeremiah is a prime example (see further \in{§}{}[subsec:jeremiah] below). Yet the consistency with which the \MT{} of late antiquity and beyond bears the general shape of the conservative type of text raises the question of how putative exceptional cases came to be. Moreover, the internal evidence is not unequivocal in my view. For these reasons, I operate here on the hypothesis that M represents the conservative text, at least on most points, for the entire \HB{} canon.\par
By the same token, there are, as stated above, historical reasons not to think the medieval \MT{} is in all respects pristine. Indeed, there are various cases where textual error in the \MT{} is all but certain. Examples include, but are not limited to, \Heb{וְדֺדָנִים} for \Heb{וְרֺדָנִים} in Gen 10:4,\footnote{Historical considerations (see Ronald S. Hendel, \italic{The Text of Genesis 1–11:} \italic{Textual Studies and Critical Edition} [Oxford: Oxford University Press, 1998], 7), the parallel in 1 Chr 1:7 (there some witnesses agree with Gen 10:4 \MT{}, but the external support for such is weak), and existence of ancient external support all argue strongly for \Heb{ורדנים}. (The small number of medieval \MSS{} that so read are, however, most likely due to polygenesis; cf. Tov, \italic{The Text-Critical Use of the Septuagint in Biblical Research}, 3\textsuperscript{rd} ed. [Winona Lake, IN: Eisenbrauns, 2015], 90–91). The rival reading, meanwhile, is readily explainable as visual error (viz. \italic{daleth/resh} confusion).} \Heb{הַאִם} for \Heb{הֲתֺם} in Num 17:28,\footnote{Here linguistic factors and the similarity of \italic{aleph} and \italic{taw} in some ancient scripts come together to argue strongly for the non-\MT{} reading (Shemaryahu Talmon, \quotation{The Paleo-Hebrew Alphabet and Biblical Textual Criticism,} in \italic{Text and Canon of the Hebrew Bible: Collected Studies} [Winona Lake, IN: Eisenbrauns, 2010], 125–170, esp. 145–146).} \Heb{וּבִזְיוֹתְיָה} for \Heb{וּבְנוֹתֶיה} in Josh 15:28,\footnote{Given the graphic similarity of the two forms, the M reading could easily have resulted from a misreading of the text, especially by a scribe taxed by the array of relatively obscure place names in this passage. There seems to be no word or place name \Heb{בזיותיה}*, while the G reading apparently has an echo in Neh 11:27 (G lacks part of that verse, but even if the minus there is accepted, such a decision would eliminate assimilation to Neh 11:27 as an explanation for the G reading in Josh 15:28) and in any case makes excellent sense here} the absence of \Heb{הַבָּאִים} in Jer 31:38,\footnote{The full phrase \Heb{הנה ימים באים} occurs twenty-one times elsewhere in the HB (fourteen times in Jeremiah), while a non-occurrence of \Heb{באים} is nowhere attested. The minus can easily be explained as haplography, owing either to the fact that \Heb{ימים} and \Heb{באים} end the same way or to the visual similarity of \Heb{באים} to the following \Heb{נאם} (indeed, the same error occurs, with paltry external support, in this same book in 7:32, 16:14, 19:6, 23:7, 48:12 49:2, and 51:52 according to Kennicott, \italic{Vetus Testamentum Hebraicum cum Variis Lectionibus}, 2 vols. [Oxford: Clarendon Press, 1776–1780]).} \Heb{כְּתֺם} for \Heb{בְּתֺם} in Ps 78:72,\footnote{Everywhere else where the phrase \Heb{תם לבב} is used in the \HB{} (Gen 20:5, 6, 1 Kgs 9:4, Ps 101:2), the preposition is \Heb{ב}, not \Heb{כ}. The anomaly in the MT here is manifestly due to the visual similarity of these two letters.} and \Heb{מוּמכָן} for \Heb{מְמוּכָן} in Esth 1:16.\footnote{Cf. vv. 14 and 21. Linguistic metathesis in so short a span of text seems unlikely as an explanation for \Heb{מוּמְכָן}*. I take the \italic{qere} to signify early recognition of the problem in the \MT{}.} Since neither the M branch nor the G branch is wholly superior, a choice must be made between them on a case by case basis.\par
Some of the principles that informed my selection of certain readings as better or best are detailed in what follows.\par
\startsubsection[reference=subsec:principle1]
Where the MT is not accepted, the preferred reading is almost always to be found in (at least) the OG\footnote{As it relates to the LXX or OG versions, the original text of each such translation is ordinarily the proper point of comparison, though at times secondary developments within the transmissional history of the Greek texts can be important where such developments stem (at least ultimately) from comparison with Hebrew or Aramaic sources no longer extant (for an example, see the source cited in \in{n.}{}[note:judg-10-16] below). In general, I follow the standard eclectic text of the \OG{} versions (Alfred Rahlfs and Robert Hanhart, eds., \italic{Septuaginta}, rev. ed. [Stuttgart: Deutsche Bibelgesellschaft, 2007]), or the Göttingen volumes (published by Vandenhoeck \& Ruprecht) where these are available.} or a witness from the Judean wilderness.\footnote{As an example of an exception to this rule, in Ps 54:5 I prefer \Heb{זדים} over \Heb{זרים} with certain Hebrew \MSS{} and the Targum, given the parallel in 86:14 and the ease with which \italic{daleth} and \italic{resh} can be confused.}\par
\stopsubsection
\startsubsection[reference=subsec:principle2]
I hold the extant witnesses to the text in high esteem,\footnote{Cf., for example \in{n.}{}[note:evidence-great-antiquity] above.} and observe that any given conjectural emendation tends not to find wide acceptance among textual scholars; consequently, all textual deviations from \Leningrad{} in the present edition rest on some degree of documentary support.\footnote[note:documentary-evidence]{In my view, versional evidence is documentary in nature. To be sure, retroversion is a delicate undertaking: translators often smooth over difficulties, and themselves err; dynamic translation can be opaque. Yet it would be gratuitous to assume the \italic{Vorlagen} of translators never differed from L, and when one works carefully, one can make many retroversions with a confidence that is both high and justifiable. See, for example, John Russiano Miles, \italic{Retroversion and Text Criticism: The Predictability of Syntax in an Ancient Translation from Greek to Ethiopic}, \SBLSCS{} 17 (Atlanta, GA: Scholars Press, 1985) and especially Tov, \italic{The Text-Critical Use of the Septuagint in Biblical Research}.\par
Note further that, where paraphrase, selective quotation, and the like have been accounted for, the New Testament can furnish important evidence for the \HB{} text (at times this evidence is indirect, in that it is the \OG{} to which the New Testament in many instances bears witness, as in Rom 3:14 = Ps 9:28 \OG{} [= Ps 10:7 \MT{}]). Similarly, use by books of the \HB{} (e.g., Chronicles) of earlier books also constitutes external evidence.\par
In rare cases, a word or phrase is reconstructed in part from one documented source, and in part from another. For example, in Lev 21:5 this edition reads \Heb{יקרחו}, taking from M (\Heb{יקרחה}) the first letter and from G (\Heb{תקרחו}) the last.}\par
\stopsubsection
\startsubsection[reference=subsec:principle3]
\italic{Qerey} readings are not accepted apart from certain or probable corroboration from other sources.\footnote{Degrees of probability, rather than certainty, obtain when (1) the text of a fragment is not altogether clear or (2) a translation is being retroverted (cf. \in{n.}{}[note:documentary-evidence] above).\par
Some \italic{kethib} readings are to be explained as accidental errors (such as \Heb{מצותו} in Deut 5:10, where the \OG{} preserves \Heb{מצותי}), while some are to explained as autographic readings considered too harsh for public reading (such as \Heb{תשגלנה} in Isa 13:16), and some are to be considered unexpected but legitimate forms (such as \Heb{ויישם} in Gen 24:33, on which see Joshua Blau, \italic{Phonology and Morphology of Biblical Hebrew}, LSAWS 2 [Winona Lake, IN: Eisenbrauns, 2010], 97 §3.4.3.3).}\par
\stopsubsection
\startsubsection[reference=subsec:principle4]
Constant vacillation among and between more or less distinct witnesses is generally to be avoided.\footnote{For example, the word \Heb{כוס} in Isa 57:17 and 22 could be a gloss on \Heb{קבעת}, yet the external evidence for an omission differs from verse to the other, and it presents a strained transmissional scenario to suppose different witnesses to have preserved or corrupted the original text in the same manner in such a short span of verses.}\par
\stopsubsection
\startsubsection[reference=subsec:principle5]
I take transcriptional error to be a leading cause of variation.\footnote{Confusion of similar letters (exactly which letters depending in part on the script being employed), haplography, and dittography being the most prominent items under this heading. Accidental omissions often \quotation{have no obvious visual triggers} (Drew Longacre, \quotation{A Contextualized Approach to the Hebrew Dead Sea Scrolls Containing Exodus} [PhD diss., University of Birmingham, 2015], 164) and accidental omissions in general appear to be more common than accidental additions (ibid.). Scribal interaction with damaged or illegible exemplars is an important category to bear in mind as well (see Drew Longacre, \quotation{Scribal Approaches to Damaged Manuscripts: Not Just a Modern Dilemma,} in \italic{The Dead Sea Scrolls and the Study of the Humanities. Method, Theory, Meaning: Proceedings of the Eighth Meeting of the International Organization for Qumran Studies (Munich, 4–7 August, 2013)}, \STDJ{} 125 [Leiden: Brill, 2018], 141–164).} Inasmuch as transcriptional error—being an inadvertent shift away from verbiage which, in literature, has been crafted carefully—much more often obfuscates rather than clarifies the sense,\footnote{The text-critical principle of preferring the harder of two or more readings, popular in biblical studies, has some validity, though I prefer the formulation of Maurice Robinson: \quotation{The reading that would be more difficult as a scribal creation is to be preferred} (\quotation{The Case for Byzantine Priority,} in \italic{Rethinking New Testament Textual Criticism}, ed. David Alan Black [Grand Rapids, MI: Baker Academic, 2002], 125–139, esp. 130, italics removed).} an important corollary of this principle is that generally the easier of two readings is to be preferred where plausible transcriptional explanations for the harder are available.\footnote{Examples: (1) In Deut 5:10 \Heb{מצותו} is more difficult than \Heb{מצותי}, yet there is no need to attempt to exegete the former, as \italic{waw} and \italic{yodh} are all but indistinguishable in some scribal hands; the difficult reading is difficult because it is the result of a visual error. (2) The reading \Heb{הרכם צפור} in Ps 11:1 is difficult in context, and might for that reason be preferred, yet can be derived from an easier (cf. Cat Quine, \quotation{The Bird and the Mountains: A Note on Psalm 11,} \italic{VT} 67.3 [2017], 470–479) reading \Heb{הרם כצפור} by supposing mere accident (transposition, plus errant placement of the word division). (3) In Eccl 7:27 \Heb{אמרה קהלת} might be preferred as the more difficult reading relative to \Heb{אמר הקהלת}, but the difficulty is probably insuperable, and \Heb{אמרה קהלת} is explainable as a scribal lapse wherein the word division was placed wrongly.}\par
\stopsubsection
\startsubsection[reference=subsec:principle6]
Where such transcriptional considerations are not in play, the reading which is less verbally or formally similar to a relevant word or passage than other reading is generally preferable. That is, scribes at times assimilated, whether deliberately or inadvertently, a span of text to something in the context,\footnote{For example, in 1 Chr 22:7 the \italic{kethib} \Heb{בנו} is probably an assimilation the previous verse (see Leslie C. Allen, \italic{The Greek Chronicles: The Relation of the Septuagint of I and II Chronicles to the Massoretic Text}, vol. 2: Textual Criticism, VTSup 27 [Leiden: Brill, 1974], 97).} to a parallel passage, or to certain verbal patterns.\footnote{Examples: (1) In Gen 1:9 the reading \Heb{מקום} is probably an error for \Heb{מקוה} prompted by the disparity of frequency (the former occurs 401 times in \Leningrad{}; the latter, nine times in \Leningrad{}; visual error is also possible here, though even the numerical factor just mentioned probably facilitated the misreading of the text), and (2) in 1 Kgs 15:9 the reading \Heb{מלך} after \Heb{אסא} is probably an inadvertent assimilation to fortuitous pattern whereby, in much of the verse, every second word is \Heb{מלך} (the error was likely facilitated by the visual similarity of \Heb{מלך} to \Heb{על}).}\par
\stopsubsection
\startsubsection[reference=subsec:principle7]
Evaluation of individual variation units is to be undertaken with an eye on (1) variation units of a similar nature,\footnote{For example, in 2 Sam 5:2 the form \Heb{והבי} could be understood as haplography before the following \Heb{את}, but given cases like \Heb{ויבו} in 1 Kgs 12:12 and \Heb{החטי} in 2 Kgs 13:6, unusual spelling seems preferable as an explanation.} and (2) the general character of the external sources involved (on these characterizations see \in{§}{}[sec:characterizations-of-books] below), though ultimately each variation unit must be decided on a case-by-case basis.\par
\stopsubsection
\startsubsection[reference=subsec:principle8]
Readings which seem to reflect the era, region, linguistic usage, and stylistic practices characteristic of the work (or, where appropriate, of the incorporated source material, as when the book of Kings cites earlier records) are preferable to readings which do not.\footnote{Examples: (1) In 2 Sam 20:23 \Heb{הכרתי} rather than \Heb{הכרי} is the term the original composition would have used (cf. David Toshio Tsumura, \italic{The Second Book of Samuel}, \NICOT{} [Grand Rapids, MI: Eerdmans, 2019], 287 n.~368). (2) The reading \Heb{ואתך} for M \Heb{אותך} in Ps 25:5 supports the acrostic pattern of the rest of the psalm, even as the G text does not display a tendency elsewhere to try to \quotation{correct} apparent deficiencies in acrostics (on an apparent lack of awareness of or interest in acrostics even in late periods, cf. Paul W. Gaebelein, \quotation{Psalm 34 and Other Biblical Acrostics: Evidence from the Aleppo Codex,} \italic{Maarav} 5–6 [1990], 127–143). (3) The plus of \Heb{אני} after \Heb{ידעתי} in Eccl 1:17 sits very well in the book, given the frequency of \Heb{אני} relative to the size of the book and the position of the pronoun after a verb (haplography, moreover, readily explains the minus).} This principle is somewhat moderated by the observation that the G text tends to assimilate and to create repetitive material within biblical books, such that some secondary readings might appear to bear characteristics of a given book.\par
\stopsubsection
\startsubsection[reference=subsec:principle9]
In my view \HB{} poetry does feature rhythmic phenomena,\footnote{Cf., for example, Theodore H. Robinson, \italic{The Poetry of the Old Testament} (London: Gerald Duckworth, 1947), 20–39.} but various unresolved questions preclude, at least for the present, the assignment of much weight to metrical considerations.\par
\stopsubsection
\startsubsection[reference=subsec:principle10]
Much remains to be studied concerning orthography (cf. §1.2 in the introduction of volume 1). In general, this edition's deviations from \Leningrad{} in orthographic matters occur where the spelling of a certain word in a given verse appears to be unique, or nearly so, to that \MS{}.\par
\stopsubsection
\startsubsection[reference=subsec:principle11]
Much also remains to be studied concerning the open and closed sections (cf. §1.3 in the introduction of volume 1). For the most part, where I have deviated from \Leningrad{}, it has been where that \MS{} differs from multiple editions of importance (particularly \Heb{כתר ירושלים}).\par
\stopsubsection
\startsubsection[reference=subsec:principle12]
I consider orthodox corruption (that is, intervention by scribes with the goal of safeguarding controverted dogma) to be rare in the MT (for examples see Judg 10:16\footnote[note:judg-10-16]{Cf. Natalio Fernandez Marcos, \italic{Judges}, vol. 7 of \BHQ{} (Stuttgart: Deutsche Bibelgesellschaft, 2012), 83*.} and 2 Sam 12:14).\footnote{Such is suggested by the paucity of clear positive examples of orthodox corruptions in the \MT{} and by the large number of negative examples wherein intervention could have occurred, but did not. Examples: (1) What in Gen 3:9 could be taken as implying ignorance on God's part is left untouched by the \MT{}. (2) The clause \Heb{ויראו את אלהי ישראל} in Exod 24:10 is apparently left untouched, while adapted texts or interpretive renderings occur in other sources: και ειδον τον τοπον ου εστηκει εκει ο θεος του ισραηλ in the LXX; \Heb{וחזו ית יקר אלהא דישראל} in Targum Onqelos. (3) The seeming contradiction between 2 Sam 24:1 and 1 Chr 21:1 did not prompt scribal intervention. Many more examples could be supplied. Legal and ritual, not theological, disagreements are what most exercised the Jewish sects in relation to one another (this theme is developed in Gary A. Rendsburg, \quotation{The Halakhic Letter-Rituals Define the Sect,} lecture 17 in \italic{The Dead Sea Scrolls}, The Great Courses [Chantilly, VA: The Teaching Company, 2010]). For an example of a similar conclusion, see Tov, \quotation{The Socio-Religious Setting,} 152.}\par
\stopsubsection
\stopsection
\startsection[reference=sec:characterizations-of-books, title={Characterizations of
Books}]
No two books of the \HB{} canon are altogether alike as it relates to their
histories of transmission. Here follows a very brief analysis of the
textual situation in each book of the \HB{}. (On the sequence of the books
see §2.1 in the introduction of volume 1.)\par
\startsubsection[reference=subsec:genesis, title={Genesis}]
M-Genesis is in exceptionally good condition, while the LXX and the Samaritan Pentateuch both transmit many assimilations, pluses, and other facilitating readings. The frequent use in the Pentateuch of \Heb{הוא}, of \Heb{ההוא}, and of \Heb{נער} as grammatically feminine (e.g., Genesis 3:12, 2:12, and 24:14) is perplexing.\footnote{The theory that \italic{waw}/\italic{yodh} confusion caused the first two anomalies (see, for example, Hendel, \italic{The Text of Genesis 1-11}, 43) would be more plausible if \italic{waw}/\italic{yodh} confusion was more widespread in \MT{}-Pentateuch than it appears to be. The explanation that \Heb{נער} \quotation{is … a survival of a system of orthography in which a final vowel was written defectively} (\italic{Gesenius' Hebrew Grammar}, ed. E. Kautzsch, rev. A. E. Cowley, 2\textsuperscript{nd} ed. [Oxford: Clarendon Press, 1910], 66 §17c) similarly faces a paucity of comparable cases involving other words.} Tentatively, I understand these (possibly dialectical) forms to be in the common gender (comparable to the pronoun \Heb{אני}), and take the vocalization to be identical to the masculine forms.\footnote{For a similar approach to the first two items see Steven E. Fassberg, \quotation{The Kethiv/Qere \Heb{הִוא}, Diachrony, and Dialectology,} in \italic{Diachrony in Biblical Hebrew}, ed. C. L. Miller-Naudé and Ziony Zevit, \LSAWS{} 8 (Winona Lake, IN: Eisenbrauns, 2012), 171–189.}\par
\stopsubsection
\startsubsection[reference=subsec:exodus, title={Exodus}]
In my view, M-Exodus is in exceptionally good condition. In the latter portions of the Tabernacle Accounts (viz., 35:4 and following), the LXX might reflect a very different \italic{Vorlage}, which \italic{Vorlage} might be argued to be superior to M. It is difficult, however, to formulate a plausible hypothesis according to which M preserves the better text for most of the book, then suddenly becomes markedly inferior at precisely the junction where the LXX preserves a superior text, even as the character of the Samaritan Pentateuch remains consistent throughout.\footnote{On the lack of firm evidence for mixture among Hebrew \MSS{} of Exodus at Qumran, see Longacre, \quotation{A Contextualized Approach,} 236–240.} See also the comments on Genesis above.\par
\stopsubsection
\startsubsection[reference=subsec:leviticus, title={Leviticus}]
M is in exceptionally good condition. See also the comments on Genesis above.\par
\stopsubsection
\startsubsection[reference=subsec:numbers, title={Numbers}]
M is in very good condition. See also the comments on Genesis above.\par
\stopsubsection
\startsubsection[reference=subsec:deuteronomy, title={Deuteronomy}]
Many argue for certain possible orthodox corruptions, but such is not my view.\footnote{As an example, in 32:8 many favor the apparent G reading \Heb{בְּנֵי אֱלֺהִים} over the M \Heb{בְּנֵי יִשְׂרָאֵל}. Yet the MT preserves the same or similar wording in Gen 6:2, 4, Job 1:6, 2:1, and 38:7 without complaint. Daniel I. Block, \italic{The Gods of the Nations:} \italic{A Study in Ancient Near Eastern National Theology}, 2\textsuperscript{nd} ed. [Eugene, OR: Wipf and Stock, 2013], 28–31) adduces various Jewish sources for a tradition of angelic oversight of the nations, but such traditions might be the cause, rather than an effect, of the G reading. \italic{Pace} Carmel McCarthy, \italic{Deuteronomy}, vol. 7 of \BHQ{} (Stuttgart: Deutsche Bibelgesellschaft, 2005), 141*, v. 9 makes good sense in the \MT{}: God's special relationship with Israel is the reason for his allocating territory to other nations only in connection with his beneficence toward the Jewish people. Cf. Edward J. Woods, \italic{Deuteronomy}, TOTC 5 (Downers Grove, IL: InterVarsity Press, 2011), 311–312.} In fact, I find M-Deuteronomy to have the fewest errors relative to its length of any book in the \MT{}. See also the comments on Genesis above.\footnote{Whether the Pentateuch as such would have been copied more carefully than other books has long been debated, so it is worth clarifying that work on the present edition did not proceed under a theoretical commitment to one position or another.}\par
\stopsubsection
\startsubsection[reference=subsec:joshua, title={Joshua}]
The degree to which the \OG{} reflects a \italic{Vorlage} differing from M is controversial. 4QJosh\textsuperscript{a}, though highly fragmentary, is generally thought to differ substantially from M. The debates over internal evidence for certain variation units are intricate, but suffice it to say here that I have not found reason to depart on any large scale from a general preference for M.\par
\stopsubsection
\startsubsection[reference=subsec:judges, title={Judges}]
The relationship between G and M is similar to that which obtains in most books. The most notable complication of Judges comes in assessing the minuses found in 4QJudg\textsuperscript{a} and 4GJudg\textsuperscript{b} but in the end there does not seem to be good reason to depart from the combined witness of OG and M at those points.\footnote{See especially Daniel I. Block, \italic{Judges, Ruth: An Exegetical and Theological Exposition of Holy Scripture}, \NAC{} 6 (Nashville, TN: Broadman and Holman, 1999), 254–255. For a different view, see Ulrich, \italic{The Dead Sea Scrolls}, 67–70.}\par
\stopsubsection
\startsubsection[reference=subsec:samuel, title={Samuel}]
The textual history of this book is very complex. Markedly different forms of the text appear in M, in 4QSam\textsuperscript{a}, and in the \OG{}. The last named witness has a tumultuous textual history itself. M, meanwhile, contains an unusually large number of transmissional errors; it appears to be dependent upon a damaged or carelessly-copied exemplar, and has the highest number of errors relative to its length of any book in the \MT{}. Nevertheless, although most scholars consider either the \OG{} or 4QSam\textsuperscript{a} to be the single best witness extant, this edition takes a minority position and assumes M to be the single best witness.\footnote{For example, I take the shorter G text of 1 Sam 17 and 18 not as evidence of a much expanded M text, but as an aversion to (perceived) doublets, such as that seen in Jubilees (cf. Abraham J. Berkovitz, \quotation{Missing and Misplaced? Omission and Transposition in the \italic{Book of Jubilees},} in \italic{HĀ-'ÎSH MŌSHE: Studies in Scriptural Interpretation in the Dead Sea Scrolls and Related Literature in Honor of Moshe J. Bernstein}, ed. Binyamin Y. Goldstein, Michael Segal, and George J. Brooke, \STDJ{} 122 [Leiden: Brill, 2017]; on 1 Sam 17–18 specifically, see further Stephen Pisano, \italic{Additions or Omissions in the Books of Samuel: The Significant Pluses and Minuses in the Massoretic, LXX and Qumran Texts} [Göttingen: Vandenhoeck and Ruprecht, 1984], 78–86, though I do not necessarily accept all the details of his arguments).}\par
\stopsubsection
\startsubsection[reference=subsec:kings, title={Kings}]
The textual history of the \OG{} the relationship of its \italic{Vorlage} to the \MT{} is intricate, but in my estimation, intrinsic evidence generally favors M readings.
\stopsubsection
\startsubsection[reference=subsec:jeremiah, title={Jeremiah}]
At least on quantitative terms, the relationship which most often obtains between M and G elsewhere is inverse here.\footnote{The sequence of the two texttypes also differs radically. In G the oracles against the nations are placed not, as in M, near the end of the book, but rather after 25:13, and G orders the material thus: 49:34–39, 46:1–28, 50:1–51:64, 47:1–7, 49:7–22, 49:1–6, 49:28–33, 49:23–27, 48:1–47, 25:15–45:5.} Many scholars take G to be the better text of the two, at least as regards length. I see it best to take a different view, for although there are similarities between the M text here and the G text in certain other books,\footnote{This is not to say that all internal arguments for the G text are equally strong. For example, Stipp points out a large number of words, phrases, and collocations that occur only, or almost only, in M-Jeremiah (Hermann-Josef Stipp, \italic{Studien zum Jeremiabuch: Text und Redaktion}, \FAT{} 96 [Tübingen: Mohr Siebeck, 2015], 83–126). Yet some examples are so specific that it is needless to ascribe significance to their non-occurrence elsewhere (e.g., ibid., 109, \quotation{\Heb{מות}-H + Objekt \italic{Gedalja},} under 41:2, 41:4). None of his examples link the M pluses with linguistic phenomena that certainly postdate the prophet, though admittedly there are a few cases that would fit well with a late date, such as the use of \Heb{מלכות} in 10:7, 49:34, and 52:31 (ibid., 113–114).} (1) M has, in at least most other cases, preserved a conservative type of text, suggesting that such might obtain in the present case as well; (2) a reworked \italic{Vorlage} can involve shortening of the text\footnote{Secondary shortening is seen, for example, in 1 Esdras (cf. Kristin De Troyer, \italic{Rewriting the Sacred Text} [Atlanta, GA: Society of Biblical Literature, 2003], 91–126, and Zipora Talshir, \italic{1 Esdras: A Text Critical Commentary}, \SBLSCS{} 50 [Atlanta, GA: Society of Biblical Literature, 2001, ix); cf. also n.~45.}; and (3) there appear to be various points of internal evidence that weigh against the G text here.\footnote{As examples: (1) The phrase \Heb{ולעמוד ברזל} seems to have good intrinsic support (see Shemaryahu Talmon, \quotation{An Apparently Redundant Reading in the Masoretic Text (Jer 1:18),} in \italic{Text and Canon of the Hebrew Bible: Collected Studies} [Winona Lake, IN: Eisenbrauns, 2010], 323–327), yet is missing in G (J. Gerald Janzen, \italic{Studies in the Text of Jeremiah}, \HSM{} 6 [Cambridge, MA: Harvard University Press, 1973], 119, suggests haplography in the \MS{} used by the \OG{} translator, but haplography seems unlikely here). (2) The omissions of \Heb{לוא} by G in 2:25, 5:10, and 15:7 are suspicious when within the \HB{} corpus this spelling is distinctively Jeremian; similar is the omission of \Heb{הזאתה} in 26:6, a former spelling is unattested anywhere else, even at Qumran where forms like \Heb{אתמה} and \Heb{הואה} are regularly encountered. It could be corrupt dittography (note the visual similarity of \italic{taw} to \italic{he}), but it does tend to reinforce the impression that forms distinctive of Jeremiah have been obscured by the G text. (3) In Jer 46:17 G has (if an assimilation, secondary to G tradition, to v. 2 is not hypothesized) the personal name \Heb{נְכו}ֹ after \Heb{פרעה}, whereas the pharaoh in question is probably Apries or Amasis (John Boardman, I. E. S. Edwards, N. G. L. Hammond, and E. Sollberger, eds., with the assistance of C. B. F. Walker, \italic{The Cambridge Ancient History}, vol. 3, part 2: The Assyrian and Babylonian Empires and Other States of the Near East, from the Eighth to the Sixth Centuries BC, 2\textsuperscript{nd} ed. [Cambridge: Cambridge University Press, 2003], 719).\par Furthermore, if the \OG{} for Jeremiah or its \italic{Vorlage} contained the deuterocanonical book of Baruch as well (see Frank Feder and Matthias Henze, eds., \italic{The Deuterocanonical Scriptures: Baruch/Jeremiah, Daniel (Additions), Ecclesiasticus/Ben Sira, Enoch, Esther (Additions), Ezra}, vol. 2B of \italic{THB} [Leiden: Brill, 2019], 4\textsuperscript{a}–22\textsuperscript{a}), though concerning the Peshitta note B. Albrektson, S. Dedering, and D. M. Walter, K. D. Jenner, and J. G. Veldman, eds., \italic{The Old Testament in Syriac according to the Peshiṭta Version, Part III Fasc. 2. Jeremiah – Lamentations – Epistle of Jeremiah – Epistle of Baruch – Baruch} [Leiden: Brill, 2019], 226 n.~4), such would point to a notably free approach to the text in this strand of the textual tradition.} In my view, then, M preserves a good text, with a rate of error little different from the M text in most other books.\par
\stopsubsection
\startsubsection[reference=subsec:ezekiel, title={Ezekiel}] §3.11 Ezekiel: The textual history of the OG of Ezekiel is controversial, but most likely the OG lacked a considerable amount of text, most notably 12:26–28, 32:24b–26, and 36:23c–38, and placed c. 37 after c. 39; the G text all but certainly lacked much material besides the passages just mentioned. Against the apparent trend of scholarly opinion, the present edition assigns overall superiority to M. This witness does, however, have a high rate of error (notably in bearing numerous dittographs).\par
\stopsubsection
\startsubsection[reference=subsec:isaiah, title={Isaiah}]
The OG is hard to assess given its style of translation, but seems not to differ greatly from M.\footnote{See, for example, Mirjam van der Vorm-Croughs, \italic{The Old Greek of Isaiah: An Analysis of Its Pluses and Minuses}, SBLSCS 61 (Atlanta, GA: 2014).} 1QIs\textsuperscript{a} and 1QIs\textsuperscript{b} are, aside from orthographic and other minor differences, not far from the MT.\footnote{Ulrich sees \quotation{[n]ine large insertions of text—of a sentence or a verse, or even several sentences or verses—were discovered in the MT, seven of which were not yet present in the text of 1QIsa\textsuperscript{a}, and three not yet in the LXX} (\italic{The Dead Sea Scrolls}, 129). Even granting, for the sake of argument, the originality of all of these minuses, this is not a large number of large-scale differences in a book of this size.}\par
\stopsubsection
\startsubsection[reference=subsec:minor-prophets, title={The Book of the Twelve}]
The OG is a generally literal translation, though the sequence of the prophets differs from that in M.\footnote{The order in the OG is: Hosea, Amos, Micah, Joel, Obadiah, Jonah, Nahum, Habakkuk, Zephaniah, Haggai, Zechariah, Malachi.} The absence of the final chapter of Habakkuk in 1QpHab is not taken as reflecting an earlier stage of the textual transmission of the book, for the scroll is a commentary on the overtly prophetic material.\footnote{\quotation{Continuous commentaries [from the Second Temple period] have been identified only on the prophet books} (ibid., 306).}\par
\stopsubsection
\startsubsection[reference=subsec:ruth, title={Ruth}]
M is in very good condition.\par
\stopsubsection
\startsubsection[reference=subsec:psalms, title={Psalms}]
The OG transmits a text not too far removed from M with the exception of the superscriptions and its approach to certain liturgical elements. The many Qumran \MSS{} of Psalms show considerable diversity,\footnote{A helpful survey appears in Peter W. Flint, \italic{The Dead Sea Psalms Scrolls and the Book of Psalms}, STDJ 17 (Leiden: Brill, 1997), 31–49.} but none is here taken as better overall than M. Some \MSS{}, from Qumran and elsewhere, employ stichometric arrangements of at least portions of the book. Such a layout is not here considered to be original to the text,\footnote{The very inconsistent treatment matters of layout receive in the Qumran \MSS{} (see Armin Lange and Emanuel Tov, \italic{THB:} vol. 1C: The Hebrew Bible; Writings [Leiden: Brill, 2017] 18\textsuperscript{a}–19\textsuperscript{a}) among other factors, point in this direction.} but might have been found in the \MS{} from which M descends, a factor which might explain the loss of certain lines from M.\footnote{Examples include the last line of Ps 13 and the superscription of Ps 33.} In any case, M contains a relatively high proportion of errors. (On the superscriptions, see §1.2 in the introduction of volume 1; on divisions within the Psalter, see §1.3 in the same place.)\par
\stopsubsection
\startsubsection[reference=subsec:job, title={Job}]
At times the existence or viability of variant readings is difficult to assess, owing to the many linguistic and exegetical puzzles in the book and to the paraphrastic style of the OG. The Hebrew \italic{Vorlage} of that version has large-scale interpolations at 2:9 and 42:17. The OG translator, for his part, abbreviated.\footnote{See Claude E. Cox, \quotation{Does a Shorter Hebrew Parent Text Underlie Old Greek Job?} in \italic{In the Footsteps of Sherlock Holmes: Studies in the Biblical Text in Honour of Anneli Aejmelaeus}, ed. Kristin De Troyer, T. M. Law, and M. Liljeström, \CBET{} 72 (Leuven: Peeters, 2014), 449–460.} In my view the Qumran Targum does not reflect a text substantially different from the MT. In the end, the MT is the single best witness extant, though it does appear to have suffered in transmission.\par
\stopsubsection
\startsubsection[reference=subsec:proverbs, title={Proverbs}]
The OG is highly paraphrastic. M might for its part contain some expansions (viz. 18:23–19:2, 20:14–19, 22:6, 23:23), and might have rearranged 31:25–26, but in my view it is instead the OG that is in secondary at these points. That said, M-Proverbs does bear a high rate of error.\par
\stopsubsection
\startsubsection[reference=subsec:ecclesiastes, title={Ecclesiastes}]
M-Ecclesiastes appears to be comparatively poorly transmitted. The Greek version is an extremely literal rendering of its \italic{Vorlage}, though that \italic{Vorlage} does seem to include facilitating elements.\par
\stopsubsection
\startsubsection[reference=subsec:song-of-songs, title={Song of Songs}]
One must assess the \MSS{} 4QCant\textsuperscript{a} and 4QCant\textsuperscript{b} and relate these witnesses to the overall textual history of the book, for \quotation{4QCant\textsuperscript{a} lacks 4:8–6:10, 4QCant\textsuperscript{b} 3:6–8, 4:4–7; 4QCant\textsuperscript{b} ends with 5:1. The text of 4QCant\textsuperscript{a} is closer to M than that of 4QCant\textsuperscript{b}.}\footnote{J. de Waard, P. B. Dirksen, Y. A. P. Goldman, R. Schäfer, M. Sæbø, \italic{General Introduction and Megilloth}, vol. 18 of \BHQ{} (Stuttgart: Deutsche Bibelgesellschaft, 2007).} In the case of the latter \MS{}, its two small gaps might well be accidental, and provisionally I take the loss of material after 5:1 to be due to the broken state of the \MS{} (not unusual at Qumran) rather than to abbreviation or a differing form of the text.\footnote{Apparently the only reason for the supposition that the \MS{} ended with 5:1 is the large final \italic{mem} in \Heb{דודים} (see Armin Lange and Emanuel Tov, eds., \italic{The Hebrew Bible: Writings}, vol. 1C of \italic{THB} [Leiden: Brill, 2017], 327\textsuperscript{b}; cf. Emanuel Tov, \quotation{Excerpted and Abbreviated Biblical Texts from Qumran,} \italic{Revue de Qumrân} 16.4 [1995], 581–600, esp. 592, and, on the uncertain origin/function of small and large letters, see David Marcus, \quotation{Does the \italic{Yod} of \Heb{נַפְשִׁי} in Ps 24:4 Represent a Miniscule \italic{Waw}?} \italic{Textus} 27 [2018], 122–134.)} The absence of 4:8–6:10 in 4QCant\textsuperscript{a} is, by contrast, very difficult to account for; abbreviation is possible, though far from certain, as the practice of abbreviation in Second Temple period times is not well understood.\footnote{A survey is provided in Tov, \quotation{Excerpted and Abbreviated Biblical Texts from Qumran.}} The passage in question could be secondary,\footnote{There are some internal factors that cast some doubt on the passage. For example, the toponym \Heb{לבנון} is used with the article outside this passage but without the article in it; in 4:2, the \italic{hapax} \Heb{הקצובות} is used, while in 6:6 the (slightly) more common \Heb{רחלים} is used; and \Heb{מי זאת} is in 6:10 followed by an article, but not so 3:6 and 8:5. Yet I am not aware of any evidence of this nature that is decisive.} but the evidence is quite difficult to assess. I tentatively prefer the overall text of M over that of 4QCant\textsuperscript{a}. The OG appears to be a fairly literal translation, though the picture might be skewed by revision towards an M-type Hebrew \MS{} in the transmissional history of the Greek text,\footnote{Cf. for example Julio Trebolle Barrera, \quotation{From Secondary Versions through Greek Recensions to Hebrew Editions,} in \italic{The Text of the Hebrew Bible and Its Editions: Studies in Celebration of the Fifth Centennial of the Complutensian Polyglot}, ed. Andrés Piquer Otero and Pablo A. Torijano Morales, \THBSup{} 1 (Leiden: Brill, 2016), 180–216, esp. 190.} and its Hebrew \italic{Vorlage} would appear not to have been conservative. I retain M in most places, judging it in the end to be in good condition.\par
\stopsubsection
\startsubsection[reference=subsec:lamentations, title={Lamentations}]
M is in good condition.
\stopsubsection
\startsubsection[reference=subsec:daniel, title={Daniel}]
The relationship between the principle Greek versions is controverted and uncertain, and that known as the \OG{} differs substantially from M. (The apocryphal additions in the Greek versions are later additions to the book.) M is taken as the more trustworthy base text.\footnote{On the \OG{} in particular, cf. Arie van der Kooij, \quotation{Compositions and Editions in Early Judaism. The Case of Daniel,} in \italic{The Text of the Hebrew Bible and Its Editions: Studies in Celebration of the Fifth Centennial of the Complutensian Polyglot}, ed. Andrés Piquer Otero and Pablo A. Torijano Morales, \THBSup{} 1 (Leiden: Brill, 2016), 428–448, noting that \quotation{Much of the literature of the Hellenistic period follows Greek literary forms and/or the canons of the Greek taste} (Shaye J. D. Cohen, \italic{From the Maccabees to the Mishnah}, 3\textsuperscript{rd} ed. [Louisville, KY: Westminster John Knox Press, 2014], 35).}\par
\stopsubsection
\startsubsection[reference=subsec:esther, title={Esther}]
M appears to be in exceptionally good condition. The relationship between the so-called Alpha Text and the other main Greek version is unclear, but both texts seem ultimately to derive from a \italic{Vorlage} close to M.\footnote{See, for example, Hanna Kahana, \italic{Esther: Juxtaposition of the Septuagint Translation with the Hebrew Text}, \CBET{} 40 (Leuven: Peeters, 2005), 441.} (As with Daniel, the apocryphal additions in the Greek versions are manifestly secondary.)\par
\stopsubsection
\startsubsection[reference=subsec:ezra-nehemiah, title=Ezra-Nehemiah]
Discussions concerning the composition and textual history of Ezra-Nehemiah and of 1 Esdras\footnote{This book survives in Greek translation (its English name derives from the fact that this book precedes the Greek translation of Ezra-Nehemiah, known to modern scholarship as 2 Esdras, in the \MSS{}; on the various book-name traditions, see further Feder and Henze, \italic{The Deuterocanonical Scriptures}, 426–428) and in translations of this Greek version.} and concerning the relationship between these two books is exceedingly intricate.\footnote{See, for example, Lisbeth S. Fried, ed., \italic{Was 1 Esdras First? An Investigation into the Priority and Nature of 1 Esdras}, \AIL{} 7 (Atlanta, GA: Society of Biblical Literature, 2011).} Suffice it to say here that I posit the priority of Ezra-Nehemiah. Accordingly, Ezra-Nehemiah forms the basic text of this edition, though 1 Esdras, being derivative of Ezra-Nehemiah, constitutes a witness to the text comparable to the usual ancient versions. G (=2 Esdras) contains a few puzzling minuses, perhaps lost due to oversight. Ezra-Nehemiah is in good condition, at least as far as present documentation goes.\footnote{There are numerous apparent discrepancies between the parallel sections of Ezra 2 and Neh 7; since these passages are part of the same book, however, each list offers internal, not external evidence, for textual questions in the other list.}\par
\stopsubsection
\startsubsection[reference=subsec:chronicles, title={Chronicles}]
The \OG{} is generally literal; its major minuses are generally explainable by visual error.\footnote{See Allen, \italic{The Greek Chronicles}, 2:132–137; cases harder to explain are listed in ibid., 159–160, the most prominent member of this second group being 1 Chr 1:11–16.} M has suffered significantly in transmission.\par
\stopsubsection
\stopsection
\startsubject[reference=sec:conclusiontitle={Conclusion}]
The history of the transmission of the \HB{} text seems to be one of generally careful transmission, though interpretive transmission took place as well. The conservative tradition seems to be concentrated particularly in the \MT{}, though various historical calamities, as well as the vast spans of time involved, strongly suggest the presence of various errors, mostly inadvertent, even in that strain of the text. These considerations call for a praxis which takes the \MT{} as its starting point, and emphasizes transcriptional phenomena. Of course, the unique elements and challenges of each book of the \HB{} must be recognized. The \MT{} seems to be soundest in Deuteronomy, Exodus, Esther, Leviticus, and Genesis; and weakest in Samuel, Ezekiel, Proverbs, Ecclesiastes, and Job.
\stopsubject
\page[no]
\stopEssay
\stoptext
\stopcomponent