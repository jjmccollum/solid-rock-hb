\startenvironment sr-style % Solid Rock Hebrew Bible common style environment
	%%%%%%%%%
	%DRAFT OPTIONS
	%%%%%%%%%
	
	%\showframe
	%\showgrid
	
	%%%%%%%%%
	% PAGE LAYOUT
	%%%%%%%%%
	
	%Setup the page trim size and printing options:
	\definepapersize[10-by-8][width=8in,height=10in]
	\setuppapersize[10-by-8][10-by-8]
	\setuparranging[mirrored] %mirrored layout for double-sided printing
	\setuppagenumbering[alternative=doublesided,location=] %undo default page numbering in middle of header; doublesided option will ensure that the document has an even number of pages
	%Setup the page layout:
	\setuplayout[
		width=6in, %textblock width
		height=8.75in, %texblock height (7.5in) + header height + footer height
		backspace=1in, %spine margin
		topspace=0.5in, %head margin - header height
		header=0.75in, %header height
		footer=0.5in, %footer height
		grid=yes %enable baseline grid
	]
	
	%Define an odd page break between books that omits headers and footers in blank pages
	\definepagebreak[blankpagebreak][yes,header,footer,odd]
	
	%Patch pagecolumns so that it does not add a blank page:
	\unprotect
	\def\page_col_stop_yes
	  {%Add a column only if this is not the last column on the page:
	   \ifnum\c_page_col_current<\c_page_col_n_of_columns
	      \column
	   \fi
	   \page
	   \endgroup
	 % \setupoutputroutine[\s!singlecolumn]%
	   \page_otr_command_set_vsize
	   \page_otr_command_set_hsize
	   \page
	   \endgroup}
	\protect
	
	%%%%%
	% FONTS
	%%%%%
	
	\definefontfeature[default][default][protrusion=pure] %allow punctuation and certain parts of letters to extend outside of the text block
	
	\definefontfeature[ebgaramond-normal][default][
		liga=yes, % standard ligatures
		dlig=yes, % ``Th'' ligature
		kern=yes, % enable kerning
		onum=yes, % use old-style numbers
		tnum=yes, % use tabular (uniform-width) numbers
		cv80=yes, %use alternate circumflex for Greek
		script=latn % Latin script
	]
	\definefontfeature[ebgaramond-header][default][
		liga=no, % disable standard ligatures
		dlig=no, % disable discretionary ligatures
		kern=no, % disable kerning (to accommodate increased letterspacing)
		onum=yes, % use old-style numbers
		tnum=yes, % use tabular (uniform-width) numbers
		script=latn % Latin script
	]
	\definefontfeature[ebgaramond-smallcaps][ebgaramond-normal][
		smcp=yes
	]
	\definefontfeature[keteryg-normal][hebrew]
	\definefontfeature[keteryg-header][hebrew][
		kern=no % disable kerning (to accommodate increased letterspacing)
	]
	\definefontfeature[keteraramtsova-title][hebrew][
		kern=no % disable kerning (to accommodate increased letterspacing)
	]
	\definefontfeature[superscript][sups=yes] %superscript
	
	\definecharacterkerning
		[essaytitlekerning]
		[factor=0.3333,
		features=letterspacing]
		
	\definecharacterkerning
		[booktitlekerning]
		[factor=1.0,
		features=letterspacing]
		
	\definecharacterkerning
		[headerkerning]
		[factor=0.25,
		features=letterspacing]
		
	%Define a macro that uses the actual superscript font feature:
	\define[1]\textsuperscript{{\feature[+][superscript]#1}}
	
	%Use EB Garamond for English and Greek text
	\starttypescriptcollection[ebgaramond]
		\starttypescript[serif][ebgaramond]
			\definefontsynonym[Serif][file:../fonts/EB-Garamond/EBGaramond12-regular.otf][features=ebgaramond-normal]
			\definefontsynonym[SerifItalic][file:../fonts/EB-Garamond/EBGaramond12-italic.otf][features=ebgaramond-normal]
			\definefontsynonym[SerifBold][file:../fonts/EB-Garamond/EBGaramond-SemiBold.otf][features=ebgaramond-normal]
			\definefontsynonym[SerifBoldItalic][file:../fonts/EB-Garamond/EBGaramond-SemiBoldItalic.otf][features=ebgaramond-normal]
			\definefontsynonym[SerifCaps][Serif][features=ebgaramond-smallcaps]
		\stoptypescript
	
		\starttypescript[ebgaramond]
			\definetypeface[ebgaramond][rm][serif][ebgaramond][default]
		\stoptypescript
	\stoptypescriptcollection
	
	%Use Apparatus SIL for text-critical notation
	\starttypescriptcollection[apparatus]
		\starttypescript[serif][apparatus]
			\definefontsynonym[Serif][file:../fonts/Apparatus-SIL/AppSILR.ttf][features=default]
			\definefontsynonym[SerifItalic][file:../fonts/Apparatus-SIL/AppSILI.ttf][features=default]
			\definefontsynonym[SerifBold][file:../fonts/Apparatus-SIL/AppSILB.ttf][features=default]
			\definefontsynonym[SerifBoldItalic][file:../fonts/Apparatus-SIL/AppSILBI.ttf][features=default]
		\stoptypescript
	
		\starttypescript[apparatus]
			\definetypeface[apparatus][rm][serif][apparatus][default]
		\stoptypescript
	\stoptypescriptcollection
	
	%Use Keter YG for Hebrew text
	\starttypescriptcollection[keteryg]
		\starttypescript[serif][keteryg]
			\definefontsynonym[Serif][file:../fonts/KeterYG/KeterYG-Medium.ttf][features=keteryg-normal]
			\definefontsynonym[SerifItalic][Serif] % we don't use this
			\definefontsynonym[SerifBold][Serif] % we don't use this
			\definefontsynonym[SerifBoldItalic][Serif] % we don't use this
		\stoptypescript
	
		\starttypescript[keteryg]
			\definetypeface[keteryg][rm][serif][keteryg][default]
		\stoptypescript
	\stoptypescriptcollection
	
	%Use Keter Aram Tsova for Hebrew titles
	\starttypescriptcollection[keteraramtsova]
		\starttypescript[serif][keteraramtsova]
			\definefontsynonym[Serif][file:../fonts/KeterAramTsova/KeterAramTsova.ttf][features=keteraramtsova-title]
			\definefontsynonym[SerifItalic][Serif] % we don't use this
			\definefontsynonym[SerifBold][Serif] % we don't use this
			\definefontsynonym[SerifBoldItalic][Serif] % we don't use this
		\stoptypescript
	
		\starttypescript[keteraramtsova]
			\definetypeface[keteraramtsova][rm][serif][keteraramtsova][default]
		\stoptypescript
	\stoptypescriptcollection
	
	%Install these typescripts:
	\usetypescript[ebgaramond]
	\usetypescript[apparatus]
	\usetypescript[keteryg]
	\usetypescript[keteraramtsova]
	
	% Define custom size scales for different text font sizes:
	\definebodyfontenvironment [12pt][xx=9pt, x=10pt, d=24pt]
	\definebodyfontenvironment [13pt][x=11pt, d=30pt]
	
	%%%%%%%
	% TEXT LAYOUT
	%%%%%%%
	
	%Setup the two-column layout for Hebrew text:
	\definepagecolumns [hebrew] [
		n=2, % number of columns
		distance=0.375in, % space between columns
		direction=reverse, % ordering of columns (RTL for Hebrew text)
		setups=hebrew:layout % use layout settings defined below
	]
	
	%Layout settings for Hebrew text:
	\startsetups[hebrew:layout]
		\setupbodyfont[keteryg, 13pt]
		\setupalign[r2l,flushleft,nothyphenated] % Hebrew should be set right-to-left, flush left, with no hyphenation
	\stopsetups
	
	%Setup alignment for the text:
	\setupalign[hanging,hyphenated] % by default, justified and hyphenated, with protrusion allowed
	
	%Setup font size and leading for text:
	\setupinterlinespace[18bp] % text line spacing
	\setupbodyfont[keteryg, 13pt] % default body font
	
	%Setup text paragraph indentation:
	\setupindenting[yes, 1em]
	
	%Ensure that whitespace respects the grid layout:
	\setupblank[line,fixed]
	
	%Modify the layout for the itemize environment:
	\setupitemize[each][packed][width=1em]
	
	%Modify the settings for the blockquote environment:
	\setupdelimitedtext[blockquote][
		leftmargin=2em,
		rightmargin=2em,
		spacebefore=line, %add one line of space before
		spaceafter=line, %add one line of space after
		indenting=no, %don't indent the first paragraph of the blockquote
		style=\tfx
	]
	
	%Define a macro that switches to inline Hebrew text:
	\define[1]\Heb{{\textdir TRT\switchtobodyfont[keteryg]#1}}
	
	%%%%%%%%%%%
	% FOOTNOTE LAYOUT
	%%%%%%%%%%%
	
	%Setup font size and leading for footnotes:
	\startsetups[footnotesetup]
		\setupinterlinespace[12bp] % footnote line spacing
		\setupbodyfont[garamond,10pt] % footnote font size
		\setupalign[normal,hanging,hyphenated] % alignment of text (left-to-right, justified, protrusion allowed, hyphenation allowed)
	\stopsetups
	
	%Setup layout of footnote text blocks:
	\setupnote[footnote][
		rule=off, %disable line separating footnotes from text
		before={\blank[18bp]}, %enforce one line of blank space between the text and the footnote block
		textcommand=\textsuperscript, %command for typesetting the footnote number in the main text
		setups=footnotesetup %setup for footnote font, leading, and alignment
	]
	%Setup formatting of footnote mark at bottom of text:
	\setupnotation[footnote][
		alternative=left, % align footnote symbol with edge of text
		hang=fit, % multi-line footnotes are not indented
		numbercommand=\textsuperscript % command for typesetting the footnote number in the footnote block
	]
	
	%Setup font size and leading for footnotes:
	\startsetups[footnotesetup]
		\setupinterlinespace[12bp] % footnote line spacing
		\setupbodyfont[ebgaramond,10pt] % default footnote font
		\setupalign[normal,hanging,hyphenated] % alignment (by default, justified and hyphenated, with protrusion allowed)
	\stopsetups
	
	%Setup layout of footnote text blocks:
	\setupnote[footnote][
		rule=off, %disable line separating footnotes from text
		before={\blank[18bp]}, %enforce one line of blank space between the text and the footnote block
		textcommand=\textsuperscript, %command for typesetting the footnote number in the main text
		setups=footnotesetup %setup for footnote font, leading, and alignment
	]
	%Setup formatting of footnote mark at bottom of text:
	\setupnotation[footnote][
		alternative=left, %align footnote symbol with edge of text (use right for RTL)
		hang=fit, %multi-line footnotes are not indented
		numbercommand=\textsuperscript %command for typesetting the footnote number in the footnote block
	]
	
	%%%%%%%%%%%%
	% APPARATUS NOTATION
	%%%%%%%%%%%%
	
	%Setup a parent class of footnotes for all apparatus entries:
	\definenote[app]
	\setupnote[app][
		rule=off, %disable line separating the apparatus from the text
		before={\blank[18bp]}, %enforce one line of blank space between the text and the apparatus
		textcommand=, %disable the default notation for the footnote symbol in the text; this will redefined for each footnote
		setups=footnotesetup %setup for footnote font, leading, and alignment
	]
	\setupnotation[app][
		numbercommand=, %disable the default notation for the footnote number; this will redefined for each footnote
		counter=, %disable the default counter for the footnote number; this will redefined for each footnote
		alternative=left, %align footnote symbol with edge of text (use right for RTL)
		hang=fit, %multi-line footnotes are not indented
	]
	
	%Define a variable containing the current chapter-verse reference covered in the apparatus:
	\setevariables[app][chapter=0,verse=0]
	
	%Define macro for adding a chapter-verse reference in the apparatus if it differs from the last chapter-verse reference in the apparatus:
	\define\SetAppRef{
		%Check if the current chapter reference matches the most recent chapter reference in the apparatus:
		\doifelse{\getvariable{text}{chapter}}{\getvariable{app}{chapter}}{%
			%If it does, then check if the verse reference matches:
			\doifelse{\getvariable{text}{verse}}{\getvariable{app}{verse}}{%
				%If it does, then do nothing:
				\nospace%
			}{%
				%If it does not, then update the latest verse reference and typeset the reference:
				\setxvariables[app][verse=\getvariable{text}{verse}]%
				{\switchtobodyfont[ebgaramond, 9pt]\textdir TLT{\bf\getvariable{text}{chapter}\,:\,\getvariable{text}{verse}}}\nobreak\hspace[medium]%
			}%
		}{%
			%If it does not, then update the latest chapter and verse references and typeset the reference:
			\setxvariables[app][chapter=\getvariable{text}{chapter},verse=\getvariable{text}{verse}]%
			{\switchtobodyfont[ebgaramond, 9pt]\textdir TLT{\bf\getvariable{text}{chapter}\,:\,\getvariable{text}{verse}}}\nobreak\hspace[medium]%
		}%
	}
	
	%Define counters for different variation types:
	\definecounter[sub]
	\definecounter[add]
	\definecounter[omit]
	\definecounter[trans]
	\setcounter[sub] [0]
	\setcounter[add] [0]
	\setcounter[omit] [0]
	\setcounter[trans] [0]
	
	%Define macros for typesetting apparatus marks:
	\define[1]\StartSubMark{{\switchtobodyfont[apparatus]⸃}\doifnot{#1}{1}{{\switchtobodyfont[ebgaramond]\textsuperscript{#1}}}} %reverse symbol for RTL
	\define\StopSubMark{{\switchtobodyfont[apparatus]⸂}} %reverse symbol for RTL
	\define[1]\AddMark{{\switchtobodyfont[apparatus]⸆}\doifnot{#1}{1}{{\switchtobodyfont[ebgaramond]\textsuperscript{#1}}}}
	\define[1]\StartOmitMark{{\switchtobodyfont[apparatus]⸋}\doifnot{#1}{1}{{\switchtobodyfont[ebgaramond]\textsuperscript{#1}}}}
	\define\StopOmitMark{{\switchtobodyfont[apparatus]⸌}} %should reverse symbol for RTL, but Apparatus SIL doesn't include U2E0D
	\define[1]\StartTransMark{{\switchtobodyfont[apparatus]⸊}\doifnot{#1}{1}{{\switchtobodyfont[ebgaramond]\textsuperscript{#1}}}} %reverse symbol for RTL
	\define\StopTransMark{{\switchtobodyfont[apparatus]⸉}} %reverse symbol for RTL
	
	%Define macros for apparatus entries:
	\define[2]\Reading{ %arg 1 is reading text, arg 2 is witness list
		\doifemptyelse{#1}{%
			{\switchtobodyfont[ebgaramond, 9pt]\textdir TLT –}\nospace\nobreak\hspace[medium]{\switchtobodyfont[ebgaramond, 9pt]\textdir TLT#2}\nobreak\hspace[medium]%represent the omission with an en-dash 
		}{%
			{\switchtobodyfont[keteryg, 10pt]\textdir TRT#1}\nospace\nobreak\hspace[medium]{\switchtobodyfont[ebgaramond, 9pt]\textdir TLT#2}\nobreak\hspace[medium]%set the reading in Hebrew
		}%
	} 
	\define\PrimaryReadingSep{{\switchtobodyfont[ebgaramond, 9pt][}\nobreak\,} %reverse symbol for RTL
	\define\SecondaryReadingSep{{\switchtobodyfont[ebgaramond, 9pt]¦}\nobreak\,}
	
	%Define a macro for typesetting an apparatus:
	\define[3]\App{% arg 1 is the type of variation, arg 2 is the lemma text, arg 3 is the variant readings
		%Define the footnote symbol based on the type:
		\doif{#1}{substitution}{%
			%Update the counter:
			\incrementcounter[sub]%
			%Typeset the marking in the text:
			\StartSubMark{\rawcountervalue[sub]}\nobreak\,%
			%Then start the apparatus note:
			\startapp%
				%Typeset the marking in the apparatus, preceded by the chapter-verse reference, if needed, and followed by the variant readings:
				\SetAppRef\StartSubMark{\rawcountervalue[sub]}\nobreak\,#3%
			\stopapp%
			%Typeset the lemma in the text, followed by the closing substitution mark:
			\nobreak#2\nobreak\,\StopSubMark%
		}%
		\doif{#1}{vocalic substantive}{%
			%Update the counter:
			\incrementcounter[sub]%
			%Typeset the marking in the text:
			\StartSubMark{\rawcountervalue[sub]}\nobreak\,%
			%Then start the apparatus note:
			\startapp%
				%Typeset the marking in the apparatus, preceded by the chapter-verse reference, if needed, and followed by the variant readings:
				\SetAppRef\StartSubMark{\rawcountervalue[sub]}\nobreak\,#3%
			\stopapp%
			%Typeset the lemma in the text, followed by the closing substitution mark:
			\nobreak#2\nobreak\,\StopSubMark%
		}%
		\doif{#1}{addition}{%
			%Update the counter:
			\incrementcounter[add]%
			%Typeset the marking in the text:
			\AddMark{\rawcountervalue[add]}\nobreak\,%
			%Then start the apparatus note:
			\startapp%
				%Typeset the marking in the apparatus, preceded by the chapter-verse reference, if needed, and followed by the variant readings:
				\SetAppRef\AddMark{\rawcountervalue[add]}\nobreak\,#3%
			\stopapp%
		}%
		\doif{#1}{omission}{%
			%Update the counter:
			\incrementcounter[omit]%
			%Typeset the marking in the text:
			\StartOmitMark{\rawcountervalue[omit]}\nobreak\,%
			\startapp%
				%Typeset the marking in the apparatus, preceded by the chapter-verse reference, if needed, and followed by the variant readings:
				\SetAppRef\StartOmitMark{\rawcountervalue[omit]}\nobreak\,#3%
			\stopapp%
			%Typeset the lemma in the text, followed by the closing substitution mark:
			\nobreak#2\nobreak\,\StopOmitMark%
		}%
		\doif{#1}{transposition}{%
			%Update the counter:
			\incrementcounter[trans]%
			%Typeset the marking in the text:
			\StartTransMark{\rawcountervalue[trans]}\nobreak\,%
			%Then start the apparatus note:
			\startapp%
				%Typeset the marking in the apparatus, preceded by the chapter-verse reference, if needed, and followed by the variant readings:
				\SetAppRef\StartTransMark{\rawcountervalue[trans]}\nobreak\,#3%
			\stopapp%
			%Typeset the lemma in the text, followed by the closing substitution mark:
			\nobreak#2\nobreak\,\StopTransMark%
		}%
	}
	
	%%%%%%%%%
	% TEXT DIVISIONS
	%%%%%%%%%
	
	%Font command to be used for section headings:
	\definefont[sectionTitleFont][Serif at 12pt]
	
	%Font command to be used for Essay headings:
	\definefont[EssayTitleFont][Serif at 24pt][line=36bp]
	
	%Font command to be used for Book headings
	\definefont[BookTitleFont][keteraramtsova at 30pt][line=36bp]
	
	%Redefine the part heading:
	\setuphead[part][
		placehead=yes,%print the heading (necessary for the part heading, which isn't normally printed)
		number=no,%do not add a number to this heading
		page=blankpagebreak,%use an odd page break with no header or footer
		header=empty,%disable headers for this page
		footer=empty,%disable footers for this page
		align={flushleft, nothyphenated},%use English alignment settings, adjusted for titling
		style=EssayTitleFont, %use the titling font
		textstyle={\setcharacterkerning[essaytitlekerning]\setcharactercasing[WORD]}, % use extra letterspacing and set in all caps
		before={\phantom{*}\blank[5*line]},%add five blank lines before
		after={\page[blankpagebreak]\noindentation}%add a double page break and disable indentation of the first paragraph
	]
	
	%Redefine the section heading:
	\setuphead[section][
		placehead=yes,%print the heading
		number=yes,% include the number
		sectionsegments=section:subsection, % section numbers should be printed first, followed by subsection number (no part or chapter numbers)
		header=nomarking,%allow headers for this page
		footer=nomarking,%allow footers for this page
		align={flushleft, nothyphenated},%align flush-left and do not hyphenate
		style=sectionTitleFont,
		numberstyle={\rm}, %set number in roman at the size of the main text
		textstyle={\sc\setcharactercasing[word]}, %set title in small caps at the size of the main text
		before={\blank[line]},%add a blank line before
		after={\blank[line]},%add a blank line after
		indentnext=no%do not indent the paragraph that follows
	]
	
	%Redefine the subsection heading:
	\setuphead[subsection][
		placehead=yes,%print the heading
		number=yes,% include the number
		header=nomarking,%allow headers for this page
		footer=nomarking,%allow footers for this page
		align={flushleft, nothyphenated},%align flush-left and do not hyphenate
		style=sectionTitleFont,
		numberstyle={\rm}, %set number in roman at the size of the main text
		textstyle={\it}, %set title in italics at the size of the main text
		before={\blank[line]},%add a blank line before
		after={\blank[line]},%add a blank line after
		indentnext=no%do not indent the paragraph that follows
	]
	
	%Define a new Essay heading at the level of a part:
	\definehead[Essay][part]
	
	%Define a new Book heading at the level of a part:
	\definehead[Book][part]
	\setuphead[Book][
		placehead=yes,%print the heading (necessary for the part heading, which isn't normally printed)
		number=no,%do not add a number to this heading
		header=empty,%disable headers for this page
		footer=empty,%disable footers for this page
		align={r2l, flushright, nothyphenated},%use Hebrew alignment settings, but make flush right
		style=BookTitleFont, %use the titling font
		textstyle={\setcharacterkerning[booktitlekerning]}, % use extra letterspacing
		before={\phantom{*}\blank[5*line]},%add five blank lines before
		after={\page[blankpagebreak]\noindentation}%add a double page break
	]
	
	%Define a Subbook heading at the level of a chapter:
	\definehead[Subbook][chapter]
	\setuphead[Subbook][
		number=no,%do not add a number to this heading
		placehead=hidden,%do not typeset anything for this heading
		page=yes%always start on a new page
	]
	
	%Define custom chapter and verse markings to be used in the header:
	\definemarking[Chapter]
	\definemarking[Verse]
	
	%Define custom chapter and verse variables to be used in the apparatus:
	\setxvariables[text][chapter=,verse=]
	
	%Define macro for starting a new Chapter:
	\define[1]\Chapter{%
		%Update the chapter title and reset the verse title (the chapter will be typeset when the verse is):
		\setxvariables[text][chapter=#1,verse=0]%
		\marking[Chapter]{#1}%
	}
	%Define macro for marking a new Chapter in a variant reading (the chapter will be typeset when the verse is):
	\define[1]\RdgChapter{%
		%Update the chapter title and reset the verse title:
		\setxvariables[rdg][chapter=#1,verse=0]%
	}
	
	%Define macro for starting a new Verse:
	\define[1]\Verse{%
		%Check if the verse variable has been reset:
		\doifelse{\getvariable{text}{verse}}{0}{%
			%If it has, then typeset both the chapter and the verse:
			\marking[Verse]{#1}{\switchtobodyfont[ebgaramond, 11pt]\textdir TLT\bf \getvariable{text}{chapter}\nobreak\,:\nobreak\,#1}\nobreak\enquad%
		}{%
			%Otherwise, just typeset the verse:
			\marking[Verse]{#1}{\switchtobodyfont[ebgaramond, 11pt]\textdir TLT\bf #1}\nobreak\enquad%
		}%
		%Update the verse variable:
		\setxvariables[text][verse=#1]%
		%Reset the textual variation type counters:
		\setcounter[sub][0]%
		\setcounter[add][0]%
		\setcounter[omit][0]%
		\setcounter[trans][0]%
	}
	
	%Define macro for marking a new Verse in a variant reading:
	\define[1]\RdgVerse{%
		%Check if the verse variable has been reset:
		\doifelse{\getvariable{rdg}{verse}}{0}{%
			%If it has, then typeset both the chapter and the verse:
			{\switchtobodyfont[ebgaramond, 9pt]\textdir TLT\bf \getvariable{rdg}{chapter}\nobreak\,:\nobreak\,#1}\nobreak\enquad%
		}{%
			%Otherwise, just typeset the verse:
			\marking[Verse]{#1}{\switchtobodyfont[ebgaramond, 9pt]\textdir TLT\bf #1}\nobreak\enquad%
		}%
		%Update the verse variable:
		\setxvariables[rdg][verse=#1]%
	}
	
	%Define macro for closed section break:
	\definehspace[closedsectionspace][3em]
	\define\ClosedSection{%
		\nospace\hspace[none]\phantom{\hspace[closedsectionspace]}\nobreak%eat up any preceding space, add a space of zero width (to allow this to break from the preceding word), and then add a long non-breaking space
	}
	
	%Define macro for closed section break in a variant reading:
	\define\RdgClosedSection{} % Empty because we've decided this isn't necessary; but define it if you want a printed representation of a closed section in the apparatus
	
	%Define macro for open section break:
	\define\OpenSection{%
		\endgraf%end previous paragraph
		\blank[halfline,samepage]%non-breaking to ensure that no column begins with the open section symbol followed by a half line
		\startalignment[middle]׆\stopalignment%
		\blank[halfline,preference]%this can (and, if possible, should) be broken across a column
		\par%begin new paragraph
	}
	
	%Define macro for open section break in a variant reading:
	\define\RdgOpenSection{} % Empty because we've decided this isn't necessary; but define it if you want a printed representation of an open section in the apparatus

	%%%%%%
	% MARGINS
	%%%%%%

	%Define macro for getting a reference range:
	\define\RefRange{%
		%Check if the first chapter matches the last chapter on the page:
		\doifsamestringelse{\fetchmarking[Chapter][1][top]}{\fetchmarking[Chapter][2][bottom]}{%
			%If the chapters match, then check if the first verse matches the last verse on the page:
			\doifsamestringelse{\fetchmarking[Verse][1][top]}{\fetchmarking[Verse][2][bottom]}{%
				%If the verses match, then the page consists of a single verse; use its reference:
				\getmarking[Chapter][1][top]\,:\,\getmarking[Verse][1][top]%
			}{%
				%If the verses do not match, then use one chapter and the verse range:
				\getmarking[Chapter][1][top]\,:\,\getmarking[Verse][1][top]\,--\,\getmarking[Verse][2][bottom]%
			}%
		}{%
			%If the chapters do not match, then use the entire reference range:
			\getmarking[Chapter][1][top]\,:\,\getmarking[Verse][1][top]\,--\,\getmarking[Chapter][2][bottom]\,:\,\getmarking[Verse][2][bottom]%
		}%
	}
	
	%Define lua function to replace single spaces with en spaces:
	\startluacode
		userdata = userdata or {}
	
		function userdata.replaceHeaderSpaces(str)
			local rep = {
				[1] = { " ", " "   },
			}
			context(lpeg.replacer(rep):match(str))
		end
	\stopluacode
	%Define a macro that uses this function:
	\define[1]\ReplaceHeaderSpaces{\ctxlua{userdata.replaceHeaderSpaces([==[#1]==])}}
	
	%Setup header and footer text:
	\setupheadertexts[\hfill{\switchtobodyfont[ebgaramond, 11pt]\feature[+][ebgaramond-header]\setcharacterkerning[headerkerning]\sc\currentstructuretitle}\hfill][][][\hfill{\switchtobodyfont[ebgaramond, 11pt]\feature[+][ebgaramond-header]\setcharacterkerning[headerkerning]\sc\currentstructuretitle}] %even left, even right, odd left, odd right
	\setupfootertexts[{\switchtobodyfont[ebgaramond, 11pt]\textdir TLT\pagenumber}][][][{\switchtobodyfont[ebgaramond, 11pt]\textdir TLT\pagenumber}] %even left, even right, odd left, odd right
	
	%%%%%%%%
	% BIBLIOGRAPHY
	%%%%%%%%
	
	%Setup bibliography settings:
	\usebtxdataset[../bib/bibliography]
	\usebtxdefinitions[chicago]%use SBL style for citation rendering
	
\stopenvironment