%Draft-only options:
\showframe
\showgrid

%%%%%%%%%
% PAGE LAYOUT
%%%%%%%%%

%Setup the page trim size and printing options:
\definepapersize[10-by-8][width=8in,height=10in]
\setuppapersize[10-by-8][10-by-8]
\setuparranging [mirrored] %mirror layout for double-sided printing
\setuppagenumbering [alternative=doublesided, location=] %undo default page numbering in middle of header; doublesided option will ensure that the document has an even number of pages
%Setup the page layout:
\setuplayout[
	width=6in, %textblock width
	height=8.75in, %texblock height (7.5in) + header height + footer height
	backspace=1in, %spine margin
	topspace=0.5in, %head margin - header height
	header=0.75in, %header height
	footer=0.5in, %footer height
	grid=yes %enable baseline grid
]

%Patch pagecolumns so that it does not add a blank page:
\unprotect
\def\page_col_stop_yes
  {%Add a column only if this is not the last column on the page:
   \ifnum\c_page_col_current<\c_page_col_n_of_columns
      \column
   \fi
   \page
   \endgroup
 % \setupoutputroutine[\s!singlecolumn]%
   \page_otr_command_set_vsize
   \page_otr_command_set_hsize
   \page
   \endgroup}
\protect

%Setup the columns layout:
\definepagecolumns [hebrew] [
	n=2, %number of columns
	distance=0.375in, %space between columns
	direction=reverse, %ordering of columns (change to reverse for RTL languages)
	setups=hebrew:align
]

%\definepagecolumns doesn't appear to support an align option, so use this instead:
\startsetups[hebrew:align]
	\setupalign[r2l,flushleft,nothyphenated] %alignment of text in columns (leave blank for justified, {r2l,flushleft,nothyphenated} for Hebrew)
\stopsetups

%%%%%
% FONTS
%%%%%
\definefontfeature[ebgaramond-normal][default][
	liga=yes, % standard ligatures
	dlig=yes, % ``Th'' ligature
	kern=yes, % enable kerning
	onum=yes, % use old-style numbers
	cv80=yes, %use alternate circumflex for Greek
	script=latn % Latin script
]
\definefontfeature[ebgaramond-header][default][
	liga=no, % disable standard ligatures
	dlig=no, % disable discretionary ligatures
	kern=no, % disable kerning (to accommodate increased letterspacing)
	onum=yes, % use old-style numbers
	script=latn % Latin script
]
\definefontfeature[ebgaramond-smallcaps][ebgaramond-normal][
	smcp=yes
]
\definefontfeature[keteryg-normal][hebrew][
	dlig=yes % enable discretionary ligatures
]
\definefontfeature[keteryg-header][hebrew][
	kern=no % disable kerning (to accommodate increased letterspacing)
]
\definefontfeature[keteraramtsova-title][hebrew][
	kern=no % disable kerning (to accommodate increased letterspacing)
]
\definefontfeature[superscript][sups=yes] %superscript

\definecharacterkerning
	[bookkerning]
	[factor=0.6666,
	features=letterspacing]
	
\definecharacterkerning
	[headerkerning]
	[factor=0.6666,
	features=letterspacing]

\starttypescriptcollection[ebgaramond]
	\starttypescript[serif][ebgaramond]
		\setups[font:fallback:serif]
		\definefontsynonym[Serif][file:../fonts/EB-Garamond/EBGaramond12-regular.otf][features=ebgaramond-normal]
		\definefontsynonym[SerifItalic][file:../fonts/EB-Garamond/EBGaramond12-italic.otf][features=ebgaramond-normal]
		\definefontsynonym[SerifBold][Serif] % we don't use this
		\definefontsynonym[SerifBoldItalic][SerifItalic] % we don't use this
		\definefontsynonym[Header][Serif][features=ebgaramond-header]
		\definefontsynonym[SerifCaps][Serif][features=ebgaramond-smallcaps]
	\stoptypescript

	\starttypescript[ebgaramond]
		\definetypeface[ebgaramond][rm][serif][ebgaramond][default]
	\stoptypescript
\stoptypescriptcollection

\starttypescriptcollection[apparatus]
	\starttypescript[serif][apparatus]
		\setups[font:fallback:serif]
		\definefontsynonym[Serif][file:../fonts/Apparatus-SIL/AppSILR.ttf][features=default]
		\definefontsynonym[SerifItalic][file:../fonts/Apparatus-SIL/AppSILI.ttf][features=default]
		\definefontsynonym[SerifBold][file:../fonts/Apparatus-SIL/AppSILB.ttf][features=default]
		\definefontsynonym[SerifBoldItalic][file:../fonts/Apparatus-SIL/AppSILBI.ttf][features=default]
	\stoptypescript

	\starttypescript[apparatus]
		\definetypeface[apparatus][rm][serif][apparatus][default]
	\stoptypescript
\stoptypescriptcollection

\starttypescriptcollection[keteryg]
	\starttypescript[serif][keteryg]
		\setups[font:fallback:serif]
		\definefontsynonym[Serif][file:../fonts/KeterYG/KeterYG-Medium.ttf][features=keteryg-normal]
		\definefontsynonym[SerifItalic][Serif] % we don't use this
		\definefontsynonym[SerifBold][Serif] % we don't use this
		\definefontsynonym[SerifBoldItalic][Serif] % we don't use this
		\definefontsynonym[Header][Serif][features=keteryg-header]
	\stoptypescript

	\starttypescript[keteryg]
		\definetypeface[keteryg][rm][serif][keteryg][default]
	\stoptypescript
\stoptypescriptcollection

\starttypescriptcollection[keteraramtsova]
	\starttypescript[serif][keteraramtsova]
		\setups[font:fallback:serif]
		\definefontsynonym[Serif][file:../fonts/KeterAramTsova/KeterAramTsova.ttf][features=keteraramtsova-title]
		\definefontsynonym[SerifItalic][Serif] % we don't use this
		\definefontsynonym[SerifBold][Serif] % we don't use this
		\definefontsynonym[SerifBoldItalic][Serif] % we don't use this
	\stoptypescript

	\starttypescript[keteraramtsova]
		\definetypeface[keteraramtsova][rm][serif][keteraramtsova][default]
	\stoptypescript
\stoptypescriptcollection

%Install these typescripts for this file:
\usetypescript[ebgaramond]
\usetypescript[apparatus]
\usetypescript[keteryg]
\usetypescript[keteraramtsova]

%Setup font size and leading for text:
\setupinterlinespace[18bp] %text line spacing
\setupbodyfont[keteryg,13pt] %body font

%Setup font size and leading for footnotes:
\startsetups[footnotesetup]
	\setupinterlinespace[12bp] %footnote line spacing
	\setupbodyfont[keteryg,10pt] %footnote font size
	\setupalign[flushleft,nothyphenated] %alignment of text in columns (use text for justified, {flushright,nothyphenated} for Hebrew)
\stopsetups

%Setup layout of footnote text blocks:
\setupnote[footnote][
	rule=off, %disable line separating footnotes from text
	before={\blank[18bp]}, %enforce one line of blank space between the text and the footnote block
	textcommand=\textsuperscript, %command for typesetting the footnote number in the main text
	setups=footnotesetup %setup for footnote font, leading, and alignment
]
%Setup formatting of footnote mark at bottom of text:
\setupnotation[footnote][
	alternative=left, %align footnote symbol with edge of text (use right for RTL)
	hang=fit, %multi-line footnotes are not indented
	numbercommand=\textsuperscript %command for typesetting the footnote number in the footnote block
]

%Setup a parent class of footnotes for all apparatus entries:
\definenote[app]
\setupnote[app][
	rule=off, %disable line separating the apparatus from the text
	before={\blank[18bp]}, %enforce one line of blank space between the text and the apparatus
	textcommand=, %disable the default notation for the footnote symbol in the text; this will redefined for each footnote
	setups=footnotesetup %setup for footnote font, leading, and alignment
]
\setupnotation[app][
	numbercommand=, %disable the default notation for the footnote number; this will redefined for each footnote
	counter=, %disable the default counter for the footnote number; this will redefined for each footnote
	alternative=left, %align footnote symbol with edge of text (use right for RTL)
	hang=fit, %multi-line footnotes are not indented
]

%Define a variable containing the current chapter-verse reference covered in the apparatus:
\setevariables[app][chapter=0,verse=0]

%Define macro for adding a chapter-verse reference in the apparatus if it differs from the last chapter-verse reference in the apparatus:
\define\SetAppRef{
	%Check if the current chapter reference matches the most recent chapter reference in the apparatus:
	\doifelse{\getvariable{text}{chapter}}{\getvariable{app}{chapter}}{%
		%If it does, then check if the verse reference matches:
		\doifelse{\getvariable{text}{verse}}{\getvariable{app}{verse}}{%
			%If it does, then do nothing:
			\nospace%
		}{%
			%If it does not, then update the latest verse reference and typeset the reference:
			\setxvariables[app][verse=\getvariable{text}{verse}]%
			{\switchtobodyfont[ebgaramond, 9pt]\textdir TLT{\bf\getvariable{text}{chapter}\,:\,\getvariable{text}{verse}}}\nobreak\enquad%
		}%
	}{%
		%If it does not, then update the latest chapter and verse references and typeset the reference:
		\setxvariables[app][chapter=\getvariable{text}{chapter},verse=\getvariable{text}{verse}]%
		{\switchtobodyfont[ebgaramond, 9pt]\textdir TLT{\bf\getvariable{text}{chapter}\,:\,\getvariable{text}{verse}}}\nobreak\enquad%
	}%
}

%Define counters for different variation types:
\definecounter[sub]
\definecounter[add]
\definecounter[omit]
\definecounter[trans]
\setcounter[sub] [0] %use \setnumber if this doesn't work
\setcounter[add] [0]
\setcounter[omit] [0]
\setcounter[trans] [0]

%Define macros for typesetting apparatus marks:
\define[1]\StartSubMark{{\switchtobodyfont[apparatus]⸃}\doifnot{#1}{1}{{\switchtobodyfont[ebgaramond]\textsuperscript{#1}}}} %reverse symbol for RTL
\define\StopSubMark{{\switchtobodyfont[apparatus]⸂}} %reverse symbol for RTL
\define[1]\AddMark{{\switchtobodyfont[apparatus]⸆}\doifnot{#1}{1}{{\switchtobodyfont[ebgaramond]\textsuperscript{#1}}}}
\define[1]\StartOmitMark{{\switchtobodyfont[apparatus]⸋}\doifnot{#1}{1}{{\switchtobodyfont[ebgaramond]\textsuperscript{#1}}}}
\define\StopOmitMark{{\switchtobodyfont[apparatus]⸌}} %should reverse symbol for RTL, but Apparatus SIL doesn't include U2E0D
\define[1]\StartTransMark{\enquad{\switchtobodyfont[apparatus]⸊}\doifnot{#1}{1}{{\switchtobodyfont[ebgaramond]\textsuperscript{#1}}}} %reverse symbol for RTL
\define\StopTransMark{\nospace\nobreak\,{\switchtobodyfont[apparatus]⸉}} %reverse symbol for RTL

%Define macros for apparatus entries:
\define[2]\Reading{ %arg 1 is reading text, arg 2 is witness list
	{\switchtobodyfont[ebgaramond, 10pt]\textdir TRT#1}\nobreak\enquad{\switchtobodyfont[ebgaramond, 9pt]\textdir TLT#2}\nobreak\enquad%set in the appropriate font and direction for the language of the text 
} 
\define\PrimaryReadingSep{{\switchtobodyfont[ebgaramond, 9pt][}\nobreak\,} %reverse symbol for RTL
\define\SecondaryReadingSep{{\switchtobodyfont[ebgaramond, 9pt]¦}\nobreak\,}

%Define a macro for typesetting an apparatus:
\define[3]\App{% arg 1 is the type of variation, arg 2 is the lemma text, arg 3 is the variant readings
	%Define the footnote symbol based on the type:
	\doif{#1}{substitution}{%
		%Update the counter:
		\incrementcounter[sub]%
		%Typeset the marking in the text:
		\StartSubMark{\rawcountervalue[sub]}\nobreak\,%
		%Then start the apparatus note:
		\startapp%
			%Typeset the marking in the apparatus, preceded by the chapter-verse reference, if needed, and followed by the variant readings:
			\SetAppRef\StartSubMark{\rawcountervalue[sub]}\nobreak\,#3%
		\stopapp%
		%Typeset the lemma in the text, followed by the closing substitution mark:
		#2\nobreak\,\StopSubMark%
	}%
	\doif{#1}{addition}{%
		%Update the counter:
		\incrementcounter[add]%
		%Typeset the marking in the text:
		\AddMark{\rawcountervalue[add]}\nobreak\,%
		%Then start the apparatus note:
		\startapp%
			%Typeset the marking in the apparatus, preceded by the chapter-verse reference, if needed, and followed by the variant readings:
			\SetAppRef\AddMark{\rawcountervalue[add]}\nobreak\,#3%
		\stopapp%
	}%
	\doif{#1}{omission}{%
		%Update the counter:
		\incrementcounter[omit]%
		%Typeset the marking in the text:
		\StartOmitMark{\rawcountervalue[omit]}\nobreak\,%
		\startapp%
			%Typeset the marking in the apparatus, preceded by the chapter-verse reference, if needed, and followed by the variant readings:
			\SetAppRef\StartOmitMark{\rawcountervalue[omit]}\nobreak\,#3%
		\stopapp%
		%Typeset the lemma in the text, followed by the closing substitution mark:
		#2\nobreak\,\StopOmitMark%
	}%
	\doif{#1}{transposition}{%
		%Update the counter:
		\incrementcounter[trans]%
		%Typeset the marking in the text:
		\StarTransMark{\rawcountervalue[trans]}\nobreak\,%
		%Then start the apparatus note:
		\startapp%
			%Typeset the marking in the apparatus, preceded by the chapter-verse reference, if needed, and followed by the variant readings:
			\SetAppRef\StartTransMark{\rawcountervalue[trans]}\nobreak\,#3%
		\stopapp%
		%Typeset the lemma in the text, followed by the closing substitution mark:
		#2\nobreak\,\StopTransMark%
	}%
}

%Define custom chapter and verse markings to be used in the header:
\definemarking[Chapter]
\definemarking[Verse]

%Define custom chapter and verse variables to be used in the apparatus:
\setxvariables[text][chapter=,verse=]

%Define an odd page break between books that omits headers and footers in blank pages
\definepagebreak[blankpagebreak][yes,header,footer,odd]

%Font command to be used for Book headings
\definefont[BookTitleFont][keteraramtsova at 26pt][36bp]

%Define a Book heading at the level of a part:
\definehead[Book][part]
\setuphead[Book][
	placehead=yes,%print the heading (necessary for the part heading, which isn't normally printed)
	number=no,%do not add a number to this heading
	header=empty,%disable headers for this page
	footer=empty,%disable footers for this page
	align={r2l, flushright, nothyphenated},%use Hebrew alignment settings, but make flush right
	style=BookTitleFont,%set text style
	textstyle={\setcharacterkerning[bookkerning]},%use the command defined above to handle the typesetting
	after={\page[blankpagebreak]\noindentation}%add a double page break
]

%Define a Subbook heading at the level of a chapter:
\definehead[Subbook][chapter]
\setuphead[Subbook][
	number=no,%do not add a number to this heading
	placehead=hidden,%do not typeset anything for this heading
	page=yes%always start on a new page
]

%Define macro for setting space before a new Chapter:
\define\PreChapter{%
	\hspace[disable]\enquad\hspace[disable] %eat up any space, then add a 0.5em space
}

%Define macro for starting a new Chapter:
\define[1]\Chapter{%
	%Update the chapter title and reset the verse title:
	\setxvariables[text][chapter=#1,verse=0]%
	\marking[Chapter]{#1}%
}

%Define macro for setting space before a new Verse:
\define\PreVerse{%
	\hspace[disable]\enquad\hspace[disable] %eat up any space, then add a 0.5em space
}

%Define macro for starting a new Verse:
\define[1]\Verse{%
	%Check if the verse variable has been reset:
	\doifelse{\getvariable{text}{verse}}{0}{%
		%If it has, then typeset both the chapter and the verse:
		\marking[Verse]{#1}{\switchtobodyfont[ebgaramond, 11pt]\textdir TLT\bf \getvariable{text}{chapter}\nobreak\,:\nobreak\,#1}\nobreak\enquad%
	}{%
		%Otherwise, just typeset the verse:
		\marking[Verse]{#1}{\switchtobodyfont[ebgaramond, 11pt]\textdir TLT\bf #1}\nobreak\enquad%
	}%
	%Update the verse variable:
	\setxvariables[text][verse=#1]%
	%Reset the textual variation type counters:
	\setcounter[sub][0]%
	\setcounter[add][0]%
	\setcounter[omit][0]%
	\setcounter[trans][0]%
}

%Define macro for closed section break:
\definehspace[closedsectionspace][3em]
\define\ClosedSection{%
	\nospace\hspace[closedsectionspace]% eat up any preceding space and add a long space
}
%Define macro for open section break:
\define\OpenSection{%
	\par\blank[halfline]\startalignment[middle]׆\stopalignment\par\blank[halfline]%
}

%Define macro for getting a reference range:
\define\RefRange{%
	%Check if the first chapter matches the last chapter on the page:
	\doifsamestringelse{\fetchmarking[Chapter][1][top]}{\fetchmarking[Chapter][2][bottom]}{%
		%If the chapters match, then check if the first verse matches the last verse on the page:
		\doifsamestringelse{\fetchmarking[Verse][1][top]}{\fetchmarking[Verse][2][bottom]}{%
			%If the verses match, then the page consists of a single verse; use its reference:
			\getmarking[Chapter][1][top]\,:\,\getmarking[Verse][1][top]%
		}{%
			%If the verses do not match, then use one chapter and the verse range:
			\getmarking[Chapter][1][top]\,:\,\getmarking[Verse][1][top]\,--\,\getmarking[Verse][2][bottom]%
		}%
	}{%
		%If the chapters do not match, then use the entire reference range:
		\getmarking[Chapter][1][top]\,:\,\getmarking[Verse][1][top]\,--\,\getmarking[Chapter][2][bottom]\,:\,\getmarking[Verse][2][bottom]%
	}%
}

%Define lua function to replace single spaces with en spaces:
\startluacode
	userdata = userdata or {}

	function userdata.replaceHeaderSpaces(str)
		local rep = {
			[1] = { " ", " "   },
		}
		context(lpeg.replacer(rep):match(str))
	end
\stopluacode
%Define a macro that uses this function:
\define[1]\ReplaceHeaderSpaces{\ctxlua{userdata.replaceHeaderSpaces([==[#1]==])}}

%Setup header and footer text:
\setupheadertexts[\rlap{\switchtobodyfont[ebgaramond, 11pt]\textdir TLT\RefRange}\hfill {\switchtobodyfont[keteryg, 13pt]\kerncharacters[0.25]\textdir TRT\ReplaceHeaderSpaces{\structureuservariable{booktitle}}} \hfill][][][\hfill {\switchtobodyfont[keteryg, 13pt]\kerncharacters[0.25]\textdir TRT\ReplaceHeaderSpaces{\structureuservariable{booktitle}}} \hfill \llap{\switchtobodyfont[ebgaramond, 11pt]\textdir TLT\RefRange}] %even left, even right, odd left, odd right
\setupfootertexts[{\switchtobodyfont[ebgaramond, 11pt]\textdir TLT\pagenumber}][][][{\switchtobodyfont[ebgaramond, 11pt]\textdir TLT\pagenumber}] %even left, even right, odd left, odd right

%Ensure that whitespace respects the grid layout:
\setupblank[line,fixed]

\starttext
\startBook[title={תהלים}][booktitle={תהלים}]
\startpagecolumns[hebrew]
	\Chapter{16}
	\Verse{11}תּוֹדִיעֵנִי אֹרַח חַיִּים שֹׂבַע שְׂמָחוֹת אֶת פָּנֶיךָ נְעִמוֹת בִּימִינְךָ נֶצַח \OpenSection 
	\Chapter{33}
	\Verse{2}הוֹדוּ לְיַהְוֶה בְּכִנּוֹר בְּנֵבֶל עָשׂוֹר זַמְּרוּ לוֹ 
	\Chapter{60}
	\Verse{6}נָתַתָּה לִּירֵאֶיךָ נֵּס לְהִתְנוֹסֵס מִפְּנֵי קֹשֶׁט סֶלָה 
	\Chapter{80}
	\Verse{1}לַמְנַצֵּחַ אֶל שֹׁשַׁנִּים עֵדוּת לְאָסָף מִזְמוֹר 
	\Chapter{92}
	\Verse{4}עֲלֵי עָשׂוֹר וַעֲלֵי נָבֶל עֲלֵי הִגָּיוֹן בְּכִנּוֹר \par
	מֹשֶׁה
\stoppagecolumns
\stopBook
\stoptext