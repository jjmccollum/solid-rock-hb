\environment ../sty/sr-style
\startcomponent
\product ../main/main
\starttext
%\startfrontmatter %uncomment to trigger appropriate conditional formatting for standalone document
\startEssay[title={Introduction}]
Millennia separate the present era from the times which saw the composition of the books that make up the Old Testament, otherwise known as the \infull{HB} (hereafter \HB{}). It is no surprise, then, that witnesses to the text of the HB—whether manuscripts (hereafter \MSS{}; singular \MS{}), ancient versions, scriptural citations in other literature, or other sources of textual information—fail fully to agree.\footnote{Inspiration and its entailments are qualities which inhere in the wording of Scripture itself (2 Tim 3:16–17), not in copying processes or translation efforts (so also the Chicago Statement on Biblical Inerrancy, Article X, reprinted in Norman L. Geisler, ed., \italic{Inerrancy} (Grand Rapids, MI: Zondervan, 1980), 496.} Prior to the advent of modern printing technology, and particularly of modern computer technology,\footnote{\quotation{The earliest printers were actually less accurate than the best scriptoria,} according to Martin Worthington, \italic{Principles of Akkadian Textual Criticism}, \SANER{} 1 (Berlin: De Gruyter, 2012), 17 n. 58; he cites in support of this claim Paul Saenger, \quotation{Colard Mansion and the Evolution of the Printed Book,} \italic{The Library Quarterly} 45.4 (1975): 405–418.} exact replication of a large segment of text was a difficult proposition, such that various copyist errors can be expected to have beset textual transmission. In addition to such oversights, occasional reworkings of the text took place, for dogmatic, stylistic, or other reasons. All these factors lead to legitimate questions concerning the precise wording of the \HB{}.\par
The textual critic assigns himself or herself the task of seeking answers to these questions.\footnote{The word \italic{critic} here is to be understood in the sense of critical thinking rather than of disapproval; the term should call to mind that evidence-based work, rather than mere opinion or unguided intuitions is ideally to inform decisions regarding the original shape of the \HB{} text.}
Observing differences among the witnesses in the content or in the extent of the text in a given scriptural passage, he or she sets out to weigh the probabilities that obtain in each case and to identify which of two or more competing readings is most likely to be the original reading.\footnote{In light of differences among and between various witnesses to certain biblical books (e.g., Jeremiah) which, due to their specific content and/or interrelatedness, could only have come about through deliberate editorial activity in ancient times, verbal differences in certain parallel passages within the HB, and other factors, it has become increasingly popular to eschew preferring one form of the text over another.
Yet my view is that large-scale emendation of a text, whatever the literary qualities thereof, does not generate a text equally as valid as the text from which the reviser worked. 
Granting that many biblical authors made use of earlier source material, the biblical books in my view originated at a particular point in history; evolution or related terms are inapt descriptions of the origins of Scripture. 
(For a similar conclusion in a related discipline, see Steve Reece, \quotation{Homer's \italic{Iliad} and \italic{Odyssey}: From Oral Performance to Written Text,} in \italic{New Directions in Oral Theory: Essays on Ancient and Medieval Literatures}, ed. Mark C. Amodio, \MRTS{} 287 [Tempe, AZ: Arizona Center for Medieval and Renaissance Studies, 2005], 43–89.)
I do, of course, grant that pursuit of the original text is a difficult task and that there is great value in mining secondary transmissional developments for historical, sociological, linguistic and other data.}
Ideally, such research should undergird an edition of the \HB{} in which the whole scriptural text can be read in sequence.
Exegesis, translation, apologetics, and, above all, Christian worship and ethics will then have a firmer basis than would be the case in the absence of text-critical analysis.\par
Presently there is a need for editions of the \HB{} informed by such textual analysis.\footnote{Most notable among single-volume editions are David Ginsburg, \Heb{תורה נביאים כתובים והברית החדשה} (1894; repr., London: Trinitarian Bible Society, 1998), \italic{Biblia Hebraica Stuttgartensia}, 5\textsuperscript{th} ed. (Stuttgart: Deutsche Bibelgesellschaft, 1997), hereafter abbreviated \BHS{}; and Yosef Ofer, ed., \Heb{כתר ירושלים תנ״ך האוניברסיתה בעברית בירושלים} (Jerusalem: N. Ben-Zvi Publishing, 2002).\par
A notable electronic edition is Mechon Mamre, \Heb{תורה נביאים וכתובים בכתב המסורה מנוקד לפי הכתר וכתבי היד הקרובים לו}, released 2009, \hyphenatedurl{http://www.mechon-mamre.org/i/t/t0.htm}, which edition is very similar to \Heb{כתר ירושלים}.\par
Most notable among multi-volume editions are \italic{Biblia Hebraica Quinta} (hereafter \BHQ{}, published by Deutsche Bibelgesellschaft) and \italic{The Hebrew Bible: An Eclectic Edition} (published by SBL Press), though both are at present incomplete.\par
Lists of textual variation and/or recommendations regarding the same have also been published, e.g. Samuel Davidson, \italic{The Hebrew Text of the Old Testament, Revised from Critical Sources} (London: Samuel Bagster, 1855).}
Nearly all editions in wide circulation today are based largely or exclusively on a single manuscript. This state of affairs is out of step with the present availability of documentary evidence from a wide range of era and locales. Moreover, the editions in widest use are at best diplomatic editions, meaning that readings preferred over the base text appear only in an apparatus crowded with other information.\par
In short, a new edition of the \HB{} is needed. This edition should draw, as much as practicable, on all pertinent sources of information, and should present a single, running text. To meet this need, the \italic{Solid Rock Hebrew Bible} integrates decisions on thousands of individual units of variation within the larger \HB{} text.\footnote{In addition to the print volume, electronic files are freely available at \hyphenatedurl{https://github.com/jjmccollum/solid-rock-hb}.}\par
Note some important limits to the scope of this edition: its apparatus does not include information on manuscript, versional, or other external evidence; nor does it discuss intrinsic evidence, scribal proclivities, and the like. In other words, the \italic{Solid Rock Hebrew Bible} represents only a set of reasoned opinions regarding the original state of the text, not an explicit engagement with other related scholarship, or a means through which readers can trace their own path through the external and internal evidence.\footnote{For an introduction to the field of \HB{} textual criticism, see especially Ellis R. Brotzman and Eric L. Tully, \italic{Old Testament Textual Criticism: A Practical Introduction}, 2\textsuperscript{nd} ed. (Grand Rapids, MI: Baker Academic, 2016) and Emanuel Tov, \italic{Textual Criticism of the Hebrew Bible}, 3\textsuperscript{rd} ed. (Minneapolis, MN: Fortress Press, 2012). For a more in-depth introduction along with a wealth of bibliographic references, see Armin Lange, ed., \italic{Textual History of the Bible}, 3 vols. (Leiden: Brill, in preparation), hereafter abbreviated \THB{}. Scholarly assessments of individual variation units are widely scattered across commentaries, journals, and the like.} Note too that not all decisions can be said to have been made with equal degrees of confidence, given the difficulties inherent in interpreting the evidence at certain points. Nevertheless, it is hoped that the result is an improvement on a text largely based on a single \MS{}.\footnote{In a similar context George Melville Bolling, \italic{The External Evidence for Interpolation in Homer} (Oxford: Clarendon Press, 1925), 253, says, \quotation{I would suggest we approach it [viz. the task of textual reconstruction] in the spirit of one who prefers half a loaf to no bread.}}\par
The following introduction divides itself into two sections. The first enumerates the elements which in combination constitute the text of the \HB{}. The second section alerts the reader to various paratextual aspects of the present edition. (For an overview of the theory and praxis lying beneath the eclectic text, see the appendix in volume 2 of this work.)\par
\startsection[reference=sec:textual-matters, title={Textual Matters}]
The following are the matters with which \HB{} textual criticism concerns itself.\par
\startsubsection[reference=subsec:books, title={The Books of the Old Testament}]
The scope of the \HB{} in this editor's view is the twenty-four\footnote{Note that English versions conventionally divide a number of books which in the Hebrew-language tradition are really one book: 1 and 2 Samuel; 1 and 2 Kings; the twelve Minor Prophets (viz. Hosea, Joel, Amos, Obadiah, Jonah, Micah, Nahum, Habakkuk, Zephaniah, Haggai, Zechariah, and Malachi); Ezra and Nehemiah (hereafter referred to as Ezra-Nehemiah); and 1 and 2 Chronicles.} books accepted as canonical both by Protestant churches and by mainstream Judaism. The premises on which this view of the canon is based are three-fold: (1) the position taken by Jesus and by his apostles on this matter is authoritative; (2) lacking clear evidence to the contrary, such a position would align itself with mainstream Jewish thought of the day; and (3) mainstream Jewish thought endorses as scriptural in the sense here under consideration precisely the twenty-four books just alluded to.\footnote{See F. F. Bruce, \italic{The Canon of Scripture} (Downers Grove, IL: InterVarsity Press, 1988), 27–54. Eugene Ulrich, \italic{The Dead Sea Scrolls and the Developmental Composition of the Bible}, \VTSup{} 169 (Leiden: Brill, 2017), 307–308, makes much of the substantial number of scrolls of Jubilees and of scrolls containing 1 Enoch or something related to it, but whatever the view of those at Qumran on these matters, such a view, being associated with a sect, by no means can be safely taken as representative of contemporary Jewish practice generally.}\par
\stopsubsection
\startsubsection[reference=subsec:letters, title={Letters}]
In each of these books the textual critic must examine several categories. The most obvious of these categories is the letter: the textual critic seeks to establish precisely which letters in which sequence are original. Two points of clarification are worth making. (1) The \italic{matres lectionis}\footnote{That is, the letters \italic{he}, \italic{waw}, \italic{yodh}, and less commonly \italic{aleph} where these letters, rather than signifying a consonant, assist in identifying the presence and nature of a vowel.}— %manual line break
including the \italic{aleph} where present for diachronic reasons\footnote{See Joshua Blau, \italic{Phonology and Morphology of Biblical Hebrew}, \LSAWS{} 2 (Winona Lake, IN: Eisenbrauns, 2010), 86 §3.3.4.1.}—do fall within this category, for an original absence of these letters from the earliest \MSS{} has not been documented. Such letters, in other words, can be added, omitted, or transposed only on the basis of sufficient evidence, as is the case with other letters. (2) The superscriptions in the book of Psalms are also part of the textual critic's concern, as there is no documentary evidence that they are secondary.\footnote{On the Peshitta version of Psalms, see Ignacio Carbajosa, \italic{The Character of the Syriac Version of Psalms: A Study of Psalms 90–150 in the Peshitta}, \MPIL{} 17 (Leiden: Brill, 2008), 19, 67–69.}\par
\stopsubsection
\startsubsection[reference=subsec:divisions, title={Divisions}]
Another item for the textual critic is word division. There are no examples of Hebrew or Aramaic biblical texts forbearing from indicating word division.\footnote{The fact that there are a number of units of variation involving the presence or absence or location of word divisions is not evidence against this any more than variation in the identity and sequence of consonants is evidence that the authors or compilers had no intent with regard to these.} Thus an editor is not free to create, eliminate, or shift the placement of word divisions without warrant.\par
There are other types of division firmly embedded within the \MS{} tradition as well, viz. the so-called open and closed sections; the divisions between prophets in the book of the twelve Minor Prophets; the divisions between psalms; and the divisions between the five small psalters (viz., Pss 1–41, 42–72, 73–89, 90–106, and 107–150), that together constitute the canonical book known today as the book of Psalms.\par
\stopsubsection
\startsubsection[reference=subsec:base-text, title={Base Text}]
The base text of this edition—that is, the text which is printed wherever I have not become reasonably convinced of the superiority of an alternative reading—is the \infull{WLC} (hereafter, \WLC{}),\footnote{Generously made freely available at \hyphenatedurl{https://tanach.us/}; the version used is 4.20.} an electronic edition based on the Leningrad Codex (National Library of Russia Firkovich B19A; hereafter, \Leningrad{}), an old, well-preserved, and reliable representative of the \HB{} text. At times, there are differences between this edition and other scholarly assessments of the readings of Leningrad{}, but it is beyond the scope of the present edition to decide between these assessments; accordingly, the base text remains \WLC{} whether or not it reflects \Leningrad{} in every particular.\par
\stopsubsection
\stopsection
\startsection[reference=sec:paratextual-matters, title={Paratextual Matters}]
There are a number of matters which, in contrast to those mentioned in \in{§}{}[sec:textual-matters], merely reflect editorial preference concerning appearance, organization of content, and other ancillaries.\par
\subsection[reference=subsec:sequence-titles, title={Sequence and Titles of Books}]
In ancient and medieval times biblical books generally circulated separately; only gradually did individual \MSS{} or printed editions feature multiple or all books of the \HB{} within its bounds. Thus, the sequence of biblical books, though of interest to historical theology and to other disciplines, does not seem to be an object for the textual critic's consideration. Yet an edition—at least a print edition—must present the books in a certain order. The sequence used here (see the table of contents) follows the earliest attested sequence (b. Baba Bathra 14b).\footnote{Manuscripts and early editions that do include multiple biblical books display a variety of sequences. See, for example, Christian D. Ginsburg, \italic{Introduction to the Massoretico-Critical Edition of the Hebrew Bible} (London: Trinitiarian Bible Society, 1897), 1–8, and Armin Lange and Emanuel Tov, eds., \italic{The Hebrew Bible: Overview Articles}, vol. 1A of \THB{} (Leiden: Brill, 2016), 44\textsuperscript{b}–46\textsuperscript{a}. The order of \Leningrad{} is: Genesis, Exodus, Leviticus, Numbers, Deuteronomy, Joshua, Judges, Samuel, Kings, Isaiah, Jeremiah, Ezekiel, the Book of the Twelve, Chronicles (which book \BHS{} places at the end of the sequence), Psalms, Job, Proverbs, Ruth, Song of Songs, Ecclesiastes, Lamentations, Esther, Daniel, Ezra-Nehemiah.} I have opted to begin each new book on a fresh page.\par
Each new book begins, after a title page,\footnote{Ancient works did not have titles in quite the modern sense, though in some cases (e.g. Jer 1:1) the first word or words of a book might have been intended as a verbal identification for the work. This edition uses the titles that have become conventional in modern Hebrew.} on a left page.\footnote{In the \MSS{} a new book often begins on the same page, but with a noticeable space intervening between the two books.} The top of each page indicates the (1) title of the appropriate biblical book and (2) the first and last verses on that page. For ease of reference, the titles featured at the top of the page include such items as \Heb{שמואל ב} and \Heb{יואל}; additionally, 2 Sam 1:1, 2 Kgs 1:1, each of the prophets in the Book of the Twelve, 2 Chr 1:1, and Neh 1:1 begins on a new page.\footnote{It will be observed that an open section (see \in{§}{}[subsec:divisions] above) intervenes between 1 Sam 31:13 and 2 Sam 1:1, between 1 Chr 29:30 and 2 Chr 1:1, and between Ezra 10:44 and Neh 1:1; these sections are indicated by an inverted \italic{nun} (cf. \in{§}{}[subsec:forms-of-divisions] below). No section of any kind intervenes between 1 Kgs 22:54 and 2 Kgs 1:1.}\par
\stopsubsection
\startsubsection[reference=subsec:forms-of-letters, title={Forms of Letters}]
It is certain that many—perhaps all—books of the \HB{} were written in a script differing markedly from modern Hebrew and Aramaic scripts. While cognizance of this fact is important in assessing what sorts of transcriptional errors are likely to have occurred at various points, I take the exact representation of the members of the Hebrew/Aramaic alphabet to be a paratextual matter.\footnote{To the extent that this practice requires justification, we may cite the words of Jesus in Matt 5:18, where he makes reference to the representations of the letters current in his day.\par
Concerning the adornments that have become traditional for synagogue scrolls, see Ada Yardeni, \italic{The Book of Hebrew Script: History, Palaeography, Script Styles, Calligraphy \& Design} (New Castle, DE: Oak Knoll Press, 2002), 268–271.\par
Inasmuch as the ancient script did not have alternate forms for letters \italic{kaph}, \italic{mem}, \italic{nun}, \italic{pe}, and \italic{tsadde} where these constitute the final consonant of a word, this edition has regularized the anomalous masoretic forms \Heb{לםרבה} in Isa 9:6 and \Heb{מנ} in Job 38:1 and 40:6.} For ease of use, I have employed a modern font, though one which I intend as a tribute to those medieval scribes without whose careful work readers today would not have the entire \HB{} text in the original languages.\footnote{The font used (available at \hyphenatedurl{https://github.com/jjmccollum/Keter-YG}) is a slightly modified version of Yoram Gnat's Keter YG.}\par
\stopsubsection
\startsubsection[reference=subsec:forms-of-divisions, title={Forms of Divisions}]
It is probable that at least the earliest of the books of the \HB{} indicated word division not by an intervening empty space but by an intervening mark. However, I have taken the form of the division to be a paratextual matter, and have opted to employ empty spaces for this purpose (as in modern Hebrew and in English).\par
Similarly, I have deemed the manner in which the sectional breaks are represented to be a paratextual matter.\footnote{Scribal practice varies, but the standard masoretic representation of the sections seems to be as follows: \quotation{(1) [A]n Open Section (\Heb{פתוחה}) has two forms. (a) It begins with the full line and is indicated by the previous line being unfinished. The vacant space of the unfinished line must be that of three triliteral words. (b) If, however, the text of the previous Section fills up the last line, the next line must be left entirely blank, and the Open Section must begin \italic{a linea} with the following line. (2) The Closed Section (\Heb{סתומה}) has also two forms. (a) It is indicated by its beginning with an indented line, the previous line being either finished or unfinished … And (b) if the previous Section ends in the middle of the line, the prescribed vacant space must be left after it, and the first word or words of the Closed Section must be written at the end of the same line, so that the break is exhibited in the middle of the line} (Ginsburg, \italic{Introduction}, 9).} In this edition open sections appear as a line that is blank apart from an inverted \italic{nun} (see Genesis 1:5–6 for an example).\footnote{The insertion of the inverted \italic{nun} adapts the practice of \Heb{כתר ירושלים}, which edition in turn adapts the practice of the Aleppo Codex (Mordechai Glatzer, ed., \italic{Jerusalem Crown: The Bible of the Hebrew University of Jerusalem, Companion Volume} (Jerusalem: N. Ben-Zvi Publishing, 2002), 52\textsuperscript{b}–53\textsuperscript{a}). This diacritical device prevents ambiguity when an open section lands at the head or the foot of a column.} Closed sections appear as an empty space considerably wider than that which separates words (see Genesis 3:15–16 for an example). Neither indicator of sections is permitted to be broken across a line.\par
I have chosen to represent (1) the seam between the constituent members of the book of the Minor Prophets and (2) the seam between the five psalters (cf. \in{§}{}[subsec:divisions] above) by beginning a new page. Individual psalms are separated from neighboring psalms by the sign which elsewhere denotes an open section.\footnote{Scribal practice varies with respect to the matters mentioned in this paragraph.}\par
\stopsubsection
\startsubsection[reference=subsec:apparatus, title={The Textual Apparatus}]
For readers' convenience, many textual divergences from the \WLC{} (see \in{§}{}[subsec:base-text] above)\footnote{Excluded from the apparatus are textual divergences relating to paragraphing matters or to purely orthographic shifts (viz. differences involving the inclusion or exclusion of \italic{aleph}, \italic{he}, \italic{waw}, and \italic{yodh} where these letters are vocalic; interchanges of the letters \italic{samekh} and \italic{sin}; interchanges of \italic{he} and \italic{waw} where the third-person masculine singular suffix is being indicated; the insertion or deletion of a word divider where the difference is not meaningful (e.g., Isa 9:5); and, in Aramaic passages, interchanges of \italic{aleph} and he where the meaning is evidently not affected). For a complete list of all such textual divergences, see the spreadsheet at \hyphenatedurl{https://github.com/jjmccollum/solid-rock-hb}.} and some of the instances of diacritical divergences from the \WLC{} (see \in{§}{}[subsec:diacritical-marks] below)\footnote{All vocalic departures from the \WLC{}—apart from (1) most of those departures which concerning the Tetragrammaton and (2) all of those departures concerning the place name \quotation{Jerusalem} (on both of these matters, see \in{§}{}[subsec:diacritical-marks] below)—can be found in the spreadsheet at \hyphenatedurl{https://github.com/jjmccollum/solid-rock-hb}.} have been noted with sigla in the text, with the text reading (marked with the siglum \SR{}) and the corresponding \WLC{} reading (marked with the \Leningrad{} siglum) supplied in the bottom margin.\footnote{By way of clarification, this principle applies, on grounds of consistency and simplicity, even where the \WLC{} diverges from \Leningrad{}, as is the case in Josh 21:36–37, a passage lacking (this lack is, in my view, a secondary development in the history of the \HB{} text) in \Leningrad{} (see Ronald Hendel, \italic{Steps to a New Edition of the Hebrew Bible}, \TCST{} 10 [Atlanta, GA: SBL Press, 2016], 12), yet present in the \WLC{} (the bottom margin of v. 36 does, however, draw attention to matter in which I think the \WLC{} is incorrect). Similarly, since the \WLC{}, in contradistinction to the \MS{}, lacks diacritics (\in{§}{}[subsec:diacritical-marks]) for \italic{kethib} (see §1.3 of the appendix) forms, the bottom margin of the present edition displays \italic{kethib} forms (where these have been rejected in favor of some other reading) without any diacritics.} This edition uses the same text-critical sigla for additions, omissions, substitutions, and transpositions relative to the critical text as those found in the \italic{Solid Rock Greek New Testament},\footnote{Joey McCollum and Stephen L. Brown, eds., \italic{Solid Rock Greek New Testament: Scholar's Edition} (North Conway, NH: Solid Rock Publications, 2018).} which, in turn, are slightly adapted from the sigla used in the Nestle-Aland critical text of the New Testament.\footnote{Barbara Aland, Kurt Aland, Johannes Karavidopoulos et al., eds., \italic{Novum Testamentum Graece}, 28\textsuperscript{th} ed. (Stuttgart: Deutsche Bibelgesellschaft, 2012).} The first difference between this edition and \WLC{} in a given verse will be printed in the bottom margin following the chapter and verse reference; all subsequent differences in the same verse will be noted on subsequent lines without the chapter and verse reference. If multiple differences of the same type occur in the same verse, then their sigla in both the text and the bottom margin will be differentiated by superscript numbers starting at 2. Where a difference concerns the placement of a verse break, the verse break (in one or both editions, depending on the type of change) will be printed in one or both readings in the bottom margin. These features are illustrated in \in{Fig.}{}[fig:features].\par
\placefigure[here][fig:features]{Examples of the textual apparatus in Ezek 41–42. Note the distinct sigla used for addition (42:1, 12), omission (41:22), and substitution (everywhere else); the labels used to distinguish between different changes of the same type in the same verse (substitution in 41:22, addition in 42:12); and the inclusion of verse numbers in one or more readings where a textual difference affects their placement (e.g., at the end of 41:21).}{\externalfigure[../img/features.pdf][height=5in]}
\stopsubsection
\startsubsection[reference=subsec:diacritical-marks, title={Diacritical Marks}]
The pointing\footnote{That is, the \italic{daghesh} and \italic{raphe} (this latter sign is lacking in most modern editions, including \BHS{}); the point which by its position distinguishes \italic{shin} from \italic{sin}; and the vowel (or null-vowel) signs \italic{kamats}, \italic{pathach}, \italic{seghol} (these with or without the reduced-vowel sign), \italic{tsere}, \italic{cholam}, \italic{chiriq}, \italic{shuruq}, \italic{qubbutz}, and \italic{schwa}.} found in late \MSS{} is not part of the original text.\footnote{These signs are lacking in all ancient \MSS{}, and traces of their development can be seen even in relatively late periods.} That said, these signs furnish such abundant aid to comprehension that I have elected to retain them—or, more exactly, to retain the Tiberian tradition, mostly as that tradition finds concrete, if imperfect, expression in \Leningrad{}.\footnote{A similar approach is taken in Ronald Hendel, \italic{Steps to a New Edition}, 31–32.} The Tiberian system reflects phonological norms that at times differ from those known to the biblical authors, yet for the purposes of the present edition, this edition in general deviates from \Leningrad{} only where that \MS{} appears to have suffered from or to have transmitted a scribal oversight, or where a fairly pressing point of exegesis warrants a deviation. A few examples:\par
\startitemize
	\item The forms \Heb{ולאדם} and \Heb{לאדם} in Gen 2:20, 3:17, and 3:21 are in the present edition represented as common, not proper, nouns.\footnote{To judge from relevant verses where the consonantal text clearly shows the article, the proper name Adam is not, with the possible exceptions of 1:26 and 2:5 (but even in those places the sense is probably \quotation{humankind}), used until 4:25. True, from 2:16 to 4:25 (excepting 2:18), the Septuagint translates as Αδαμ; the Vulgate renders with Adam starting at 2:19 (either because of belated consultation of the Septuagint or because of a similar sense of what clarity would require of the translator); Targum Onqelos renders \Heb{אדם} as from 1:27 (there is of a difference of reading in 1:26); and the Peshitta renders as a name starting with 1:27 (the subject of the first verb in 3:24 is made implicit). These renderings, however, I take to be interpretive in origin. Note the divergences among them; the tendency of each of these translations toward paraphrase; and the difficulty of envisioning a scenario in which the \infull{MT} (hereafter \MT{}) of Genesis, a generally reliable witness, would add an article in such a large number of instances and with little motivation to do so.}
	\item In Hos 8:11 the first occurrence of \Heb{לחטא} is pointed as \Heb{לְחַטֵּא}.\footnote{See, e.g., Duane A. Garrett, \italic{Hosea, Joel: An Exegetical and Theological Exposition of Holy Scripture}, NAC 19A (Nashville, TN: Broadman \& Holman, 1997), 186–187. The sense is ironic (and involves wordplay): \quotation{Though Ephraim has multiplied altars to cleanse sin, they have been for him altars to incur sin.} A very similar sort of irony (hire becomes tribute) occurs in the previous verse.}
	\item In Jonah 3:2 \Leningrad{} has \Heb{וִּקְרָא} for \Heb{וּקְרָא}.\footnote{The \Leningrad{} form here is neither meaningful nor, it would seem, even traditional, but simply a mistake.}
	\item In Ps 34:6 the forms \Heb{הביטו} and \Heb{ונהרו} are here pointed as imperatives, not as indicatives.\footnote{See, e.g., Nancy deClaissé-Walford, Rolf A. Jacobson, and Beth LaNeel Tanner, \italic{The Book of Psalms}, NICOT (Grand Rapids, MI: Eerdmans, 2014), 322 n. 4.\par
I have left in place forms like \Heb{חַטֺּאות} in 1 Kgs 14:16, for while the masoretic vocalization does not fully represent the orthographic approach of the author (in this example, the \italic{waw} is meant to represent the final vowel sound of this word), (1) the policy of the present edition, as noted above, is generally to retain the pointing of \Leningrad{}; and (2) the diacritics in these places do not leave the sense in doubt.}
\stopitemize

Sometimes these vocalic changes assume the existence of lexical or other linguistic phenomena unknown (due to diachronic linguistic developments and to limited access to relevant archaeological and literary materials) to those responsible for the masoretic vocalization. Some examples:\par
\startitemize
	\item The puzzling word \Heb{שילה} in Gen 49:10 is (if tentatively) pointed \Heb{שְׂיָלֹה} and understood to mean something like prince or ruler.\footnote{Following the suggestion cited in Ludwig Koehler and Walter Baumgartner, \italic{The Hebrew and Aramaic Lexicon of the Old Testament}, translated and edited under the supervision of Mervyn E. J. Richardson, 2 vols. (Leiden: Brill, 2001; hereafter \HALOT{}), 1478\textsuperscript{b}. The passage is not addressed in Yoshiyuki Muchiki, \italic{Egyptian Proper Names and Loanwords in North-West Semitic}, \SBLDS{} 173 (Atlanta, GA: Scholars Press, 1993), but the posited phonetic correspondences seem plausible according to his data (255, 259–260; cf. Benjamin J. Noonan, \italic{Non-Semitic Loanwords in the Hebrew Bible: A Lexicon of Language Contact}, \LSAWS{} 14 [Winona Lake, IN: Eisenbrauns, 2019], 274).}
	\item The noun \Heb{ערכך} in Lev 5:15 and elsewhere is taken to be a noun with \quotation{a partially-reduplicated root,} not a noun from the root \Heb{ער״כ} with a second-person possessive suffix.\footnote{W. Randall Garr and Steven E. Fassberg, eds., \italic{A Handbook of Biblical Hebrew}, vol. 1: Periods, Corpora, and Reading Traditions (Winona Lake, IN: Eisenbrauns, 2016), 64–65.}
	\item In a number of cases an alternative morphology of the second-person feminine singular in the suffix tense-form and of the second feminine singular personal pronoun is here recognized.\footnote{Cf. Blau, \italic{Phonology and Morphology}, 161 §4.2.2.3. An example of the former is found in Jer 2:33; an example of the latter appears in Judg 17:2.}
	\item Occasionally, a \italic{lamedh} prefix is understood not as the preposition but as an asseverative or topicalizing particle.\footnote{John Huehnergard, \quotation{Asseverative *\italic{la} and Hypothetical *\italic{lu}/\italic{law} in Semitic,} \JAOS{} 103.3 (1983): 569–593.} At times, this particle is apparently spelled in a fuller manner, and thus, pointing aside, looks like the negative marker \Heb{לא}.\footnote{Cf. ibid., 590 n. 191. As examples, see Exod 6:3 (\quotation{with my name ... I indeed made myself known to them}) and 8:22 (\quotation{If we sacrifice ... will they not stone us?}). The pointing uniformly involves a \italic{kamats} (cf. the previous footnote).}
\stopitemize

For the most part, I have not attempted to reconcile various discrepancies within the text of \Leningrad{} concerning the vocalization of certain proper names.\footnote{For example, contrast the pointing of \Heb{נהלל} in Josh 21:35 with that in Judg 1:30. I have, however, altered the vocalization of Nebuchadnezzar's name throughout the text in accordance with the orthography found in Jer 49:28 and Ezra 2:1 (cf. David Marcus, \italic{Ezra and Nehemiah}, vol. 20 of \BHQ{} [Stuttgart: Deutsche Bibelgesellschaft, 2006], 39*).} That said, a few proper names call for special comment.\par
First, there is the Tetragrammaton (\Heb{יהוה}). The pointing varies in the \MS{} tradition, but at any rate manifestly differs from earlier pronunciation(s) in a way that is not merely linguistic. Granting utilization of the Tiberian system, the vocalization familiar to modern scholarship (\Heb{יַהְוֶה}) seems to be a reasonable approximation of the historical phonetic shape of the name, and—with no disrespect intended for standard practice within modern Judaism—is here adopted. An exception is Chronicles, where the tradition of substituting other words for \Heb{יהוה} might have already begun,\footnote{Note (1) the use of \Heb{יהוה אלהים} in 1 Chr 17:16 and 17 for \Heb{אדני יהוה} in 2 Sam 7:18 and 19; (2) the non-use of \Heb{אדני} with a divine referent in Chronicles; (3) the decline in use of the Tetragrammaton outside of Scripture quotations, seemingly evident even in some of the exilic and post-exilic biblical books (e.g., as Koehler and Baumgartner, \HALOT{}, 1:395\textsuperscript{b} note, Daniel); and (4) the tradition, evident in some of the early versions (though textual critics debate whether κυριος was originally the equivalent of the Tetragrammaton in the Old Greek versions) and in the New Testament, of avoiding transliteration of the Tetragrammaton, especially by using a word meaning \italic{lord}.} though even there the pointing of \Leningrad{} sees alteration.\par
Second, the place name \quotation{Jerusalem} seems to have two morphological realizations in ancient Hebrew, viz. \Heb{יְרוּשְׁלֵם} and \Heb{יְרוּשָׁלַיִם}. The \MT{} points the former as if it was the latter, but this edition maintains a distinction in its vocalization.\par
Third, the vocalization of the proper name \Heb{יששכר} which appears in \Leningrad{} to reflect a folk etymology unlikely to have been operative in biblical times; I have pointed this as \Heb{יִשְׂשָׂכָר}.\par
I have elected to exclude the accentuation system found in \Leningrad{}, which system, like the vocalization just discussed, is not found in the ancient \MSS{}. As valuable as the accentuation system is, including it in this edition would have entailed great effort to make it compatible with the present eclectic text. Most readers today, meanwhile, make little use of the signs. Logical or rhythmic pauses, accordingly, must be supplied mentally by the reader, and those interested in traditional accentuation will have to consult other sources.\par
\stopsubsection
\startsubsection[reference=subsec:other-features, title={Other Masoretic Paratextual Features}]
The masoretic tradition includes various features not yet mentioned above that I have deemed paratextual, such as the small and large letters\footnote{See Tov, \italic{Textual Criticism}, 53–54 and David Marcus, \quotation{Does the \italic{Yod} of \Heb{נַפְשִׁי} in Ps 24:4 Represent a Miniscule \italic{Waw}?} \italic{Textus} 27 (2018): 122–134.} and special layouts for certain portions of the text.\footnote{See Glatzer, \italic{Companion Volume}, 53\textsuperscript{b}–54\textsuperscript{a}. (Note that these layouts are not reproduced in \BHS{}.)} I have omitted these elements from the present edition.\par
\stopsubsection
\startsubsection[reference=subsec:chapter-verse-numbering, title={Chapter and Verse Numbering}]
Chapter and verse numbers, though not original to the \HB{} text, have been included in order to assist readers in identifying and in locating specific passages. Several remarks must be made. First, the versification system is (substantially) that of the \WLC{}, and not of any mainstream English translation. So, for example, in many psalms the English verse number is lower by one than the Hebrew verse number, for often the superscription is verse 1 in the \WLC{}, rather than a notation separate from standard versification. On rare occasions (e.g., Deut 32:5–6) there have been slight changes to the \WLC{} versification in the interest of clarity. Second, the \italic{soph pasuq} sign found in many \MSS{} (including \Leningrad{}, if inconsistently) has been omitted, the presence of Arabic numerals being sufficient indication of a verse break. Third, Pss 9 and 10 are considered here to be a single psalm (observe the lack, between Ps 9 and Ps 10, of the paratextual device by which psalms are shown to be separate),\footnote{In this respect \BHS{} does not accurately reflect \Leningrad{}, which \MS{} has two psalms here.} such that this edition has 149, not 150, psalms; for convenience, however, traditional chapter and verse numbering is retained.\par
\stopsubsection
\startsubject[title={Acknowledgments}]
I wish to thank my wife, Becky Brown, for her constant support; and my father, Laurence Brown, for teaching me to listen to God's Word, and for introducing me to the joys of textual criticism. Special thanks go to Joey McCollum, for immense assistance in many of the various technological aspects of this project.\par
\stopsubject
\startsubject[title={Conclusion}]
Given the scale of this project and my own fallibility, I expect that various oversights and errors of judgment will appear in these pages. I request that readers send notices of errata and suggestions for improvement to \goto{contact.solidrockpublications@gmail.com}[mailto:contact.solidrockpublications@gmail.com]. While a reply to every email cannot be guaranteed, any assistance provided is appreciated.\par
May this edition of the \HB{} glorify God and edify the Church, ὅσα γὰρ προεγράφη εἰς τὴν ἡμετέραν διδασκαλίαν προεγράφη, ἵνα διὰ τῆς ὑπομονῆς καὶ διὰ τῆς παρακλήσεως τῶν γραφῶν τὴν ἐλπίδα ἔχωμεν (Rom 15:4).\par
\stopsubject
\page[blankpagebreak]
\stopEssay
%\stopfrontmatter
\stoptext
\stopcomponent